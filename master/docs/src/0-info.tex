\thispagestyle{empty}
\pagestyle{empty}

\begin{center}
Reidar Conradi and Finn Olav Bj{\o}rnson (latex version {\O}yvind Hauge)
% The use of Norwegian letters is not recommended as they can be lost if the files are transferred between windows and
% unix file systems with different charater encoding. 
% See: http://omega.albany.edu:8008/Symbols.html for a list of latex symbols.

Possible template for a PhD article thesis

IDI, NTNU \today

\url{www.idi.ntnu.no/grupper/su/publ/ese/phd-thesis-temp-latex.zip} Comments appreciated;  send to
\url{reidar.conradi@idi.ntnu.no}! Please send comments on this Latex version of the template to
\url{oyvind.hauge@idi.ntnu.no}.

\end{center}

Background: This is a possible but not mandatory \textbf{template} for how to \textbf{structure and format a PhD 
thesis}, being a ''light-weight'' \textbf{article thesis, not} a full \textbf{monograph}. That is, a thesis containing 
an introduction and summary of some 50 pages, followed by a large Appendix with a verbatim copy of the selected and 
published papers. For the secondary papers, just list their authors, title, publication channel etc. and put their 
abstracts in another and much slimmer Appendix.

Then some admissions: there is \textit{no fixed and agreed-upon rule} for how many and what kind of papers (journal, 
conference, \ldots) that will suffice for an article thesis. That will depend on the traditions of the actual 
discipline, such as medicine, mathematics, sociology, or informatics. For the latter, the total number of selected 
papers will typically be 5-7. Most papers should have yourself as the prime (first) author and with at least 2 papers 
in journals or top-level conferences (say with acceptance rates less than 20\%). 1-2 of the papers may be submitted, 
but still awaiting decision wrt. publication, although they are ''ready for publication''. For all co-authored papers, 
you must briefly describe your own contribution by its share of substance and effort. Hint: Pure internal reports are 
discouraged. Medicine at NTNU also has a rule that co-authored papers cannot be included in more than two theses. Try 
to sort all this out in close dialogue with your advisor.

The rest of this page contains information regarding formatting of text and can be deleted once you've finished reading 
it. For more recent information on formatting, check the website of NTNU-trykk: \url{http://www.ntnu.no/ntnu-trykk/} 
This template is formatted according to the specifications given on this site as of 25 October 2007: Top and side 
margins must be 30 mm, bottom margin is set to 35 mm. Normal text is formatted as \textbf{Times New Roman 12pt}. The 
reason for using 12pt is that the publisher will down-scale the A4-format of your present manuscript to a B5-format 
(17\% linear scale-down or roughly 10pt) in the printed thesis.

\textcolor{red}{Red text is explanatory ''meta-''information and should be deleted from the final thesis.}

\textcolor{blue}{Blue text is filler text, usually containing later thesis contents, and needs to be replaced.}

\textcolor{green}{Green text is example text and can be deleted or changed.}

