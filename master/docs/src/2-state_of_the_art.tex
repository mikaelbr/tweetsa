\chapter{State of the Art}

In this section it is described how a systematic literature review (\nom{SLR}{Systematic Literature Review}) was conducted to establish the state of the art of a Twitter Sentiment Analysis (\nom{TSA}{Twitter Sentiment Analysis}) system, and the result of the review. The first section, Section~\ref{sec:slrintro}, covers the introduction and defines our research questions. The review method is described in Section~\ref{sec:slrmethod} and the result in Section~\ref{sec:slrresults}.

\section{Introduction}
\label{sec:slrintro}

SLRs are still new in the field of computer science. There are few examples of an SLR in practise. The method we used is heavily inspired from the documentation paper by~\cite{paper:slrdesc} and the Master's thesis by~\cite{master:slr}. \\

\noindent We defined our research questions (\nom{RQ}{Research Question}) as the following:

\begin{description}

\item[RQ1] What are some of the existing solutions for SA (sentiment analysis) in the Twitter Corpus.
\item[RQ2] How does the different solutions found by addressing RQ1 compare to each other with respect to micro-blogs like Twitter.
\item[RQ3] What is the strength of the evidence in support of the different solutions?
\item[RQ4] What implications will these findings have when creating the application/system?

\end{description}

\section{Review method}
\label{sec:slrmethod}

In this section we will describe our process step by step according to the systematic literature review. 

\subsection{Planning}

\begin{description}

	\item[1. Identification of the need for a review] \hfill \\
		Assumed that a review is needed for this project. 
		
	\item[2. Commissioning a review] \hfill \\
		Assumed to be commissioned for this project, so no commission report was produced. 

	\item[3. Specifying the research question(s)] \hfill \\
		We defined the RQs based on what we felt needed to be researched when finding the state-of-the-art for sentiment analysis systems on Twitter. The RQs can be seen in Section~\ref{sec:slrintro}. 

	\item[4. Developing a review protocol] \hfill \\
		The review protocol \nom{SLRP}{Systematic Literature Review Protocol} was developed in the early stages of the project. After the first revision it was evaluated by the project supervisor. The protocol was revised several times during the execution of the review.
	

	\item[5. Evaluating the review protocol] \hfill \\
		The protocol was evaluated by the project supervisor. 

\end{description}


\subsection{Conducting}

\begin{description}

	\item[1. Identification of research] \hfill \\
		We defined a series of keywords and synonyms to construct a search string to use. The search string was defined to find papers relevant to our research questions. The development of this search string is documented by Appendix~\ref{apx:slrp}. The search string we used was:
		
		\begin{verbatim}
		("Sentiment Analysis" OR "Sentiment Classification" 
		OR "Opinion Mining") AND (Twitter OR Microblog)
		\end{verbatim}
		
		For the search domain, we used Google Scholar. Google Scholar accumulates results from several different sources and gave many results for our search. Many of the results given corresponded to the studies handed out by the projects supervisor, a domain expert. 
		
		We limited the search to only give papers released after 2008. This is due to Twitter being as new as it is.
		
		The search resulted in 1060 papers, but after the first 8 pages of results (with 10 papers per page), we found that the papers got mostly irrelevant and we had limited resources and time for handling all 1060 papers. This resulted in a set of 80 papers, ready to go through the selection process.
		
		

	\item[2. Selection of primary studies] \hfill \\
		Firstly we defined a set of primary inclusion criteria. These criteria were applied to the title and abstract of the study. If a paper did not pass the criteria, it was dropped from the set. Secondly we defined secondary inclusion criteria. These criteria were checked against the full text of the study. These inclusion criteria are documented in the protocol in Appendix~\autoref{apx:slrp}. 
		
		After both selection processes, we had a set of 23 papers. 

	\item[3. Study quality assessment] \hfill \\
		With the help of \cite{paper:slrdesc} we defined a set of 10 quality criteria. All of the criteria are documented in the protocol in Appendix~\autoref{apx:slrp}. Each of the papers in our set was assessed with all of these criteria. If the paper met the criteria, it would get 1 point, if met partly it would get 0,5 points, and if it did not meet the criteria it would get 0 points. 
		
		The papers with the lowest score did not get taken as much into consideration when defining the state-of-the-art. 
		
	\item[4. Data extraction and monitoring] \hfill \\
		We defined a set of fields and information categories we thought were important in order to answer our research questions. These data features were used to generate a table of information. The information was retrieved by reading the papers and manually extracting the data we needed.
	

	\item[5. Data synthesis] \hfill \\
		This step was included in the data extraction step.
\end{description}

% This probably shouldn't be here but in the SLR chapter..
% For this report, the data synthesis is included as a part of the data extraction. 

\subsection{Reporting}


\begin{description}

	\item[1. Specifying dissemination strategy] \hfill \\
		As this is for a specialization project, the result of the SLR is presented in this report.  

	\item[2. Formatting the main report] \hfill \\
		The SLR was written as a section in this specialization project report. 

	\item[3. Evaluating the report] \hfill \\
		It is mandatory for an expert to review this report as well as a project supervisor, as it is a report for the specialization project.

\end{description}

\section{Results}
\label{sec:slrresults}

In this section the result of the systematic literature review is presented. In Section~\ref{sec:selected} all of the extracted data from the selected studies are presented. The assessed quality is also presented. 

In Section~\ref{sec:stateofart} the answers to our research questions are given. 

\subsection{Selected studies}
\label{sec:selected}

In this section all of the selected studies are presented in table format as a part of the data extraction step in the SLR. The results can be found in~\autoref{tab:extraction}. In addition to the data features defined in the review protocol, the total quality is added to be a part of the extraction table.


\begin{sidewaystable}
    \centering
    \caption{Data extraction step, table 1/4. Showing data as per defined in the SLRP in attachment \autoref{apx:slrp}}
    \scriptsize
    \label{tab:extraction}
    \begin{longtable}{|c|p{3cm}|p{4cm}|p{0.6cm}|p{1cm}|p{1.3cm}|p{4cm}|p{3cm}|p{0.3cm}|} 
    
    \hline
    \textbf{\#ID} & \textbf{Authors} & \textbf{Title} & \textbf{Pub. Year} & \textbf{System name} & \textbf{ML Algorithm} & \textbf{Dataset} & \textbf{Findings} \& \textbf{Conclusions} & \textbf{QA} \\ 
    \hline
    
    S1 & Hassan Saif, Yulan He \& Harith Alani & Semantic Smoothing for Twitter Sentiment Analysis & 2012 & - & NB & http://twittersentiment.appspot.com/ & Slight improvement by .3\% & 7,5 \\ \hline  
    
    S2 & Subhabrata Mukherjee, Akshat Malu, A.R. Balamurali, Pushpak Bhattacharyya & TwiSent: A Multistage System for Analyzing Sentiment in Twitter & 2012 & TwiSent & NB & C-Feel-IT dataset.  & Improvements with filtering/ normalization. Best: 66,69\% with manually annotated dataset  & 8,5 \\ \hline  
    
    S3 & Adam Bermingham \& Alan Smeaton & Classifying Sentiment in Microblogs:Is Brevity an Advantage? & 2010 & - & NB, SVM & "CLARITY dataset (removed due to Twitter terms), Pang \& Lee's movie corpus,TREC Blogs06 corpus and a collection of microreviews from Blippr" & Accuracy of 74.85\% using NBC and unigrams. Easier to classify microblogs than long texts  & 9,0 \\ \hline  
    
    S4 & Wilas Chamlertwat, Pattarasinee Bhattarakosol, Tippakorn Rungkasiri, Choochart Haruechaiyasak & Discovering Consumer Insight from Twitter via Sentiment Analysis & 2012 & MSAS & SVM, Lexical (non ML) & Self compiled, manually annotated 600 tweets & Sentiment Analysis can be useful for consumer research. No accuracy for classification.  & 8,0 \\ \hline  
    
    S5 & Dmitry Davidov, Oren Tsur \& Ari Rappoport & Enhanced Sentiment Learning Using Twitter Hashtags and Smileys & 2010 & - & HFW/CW & Dataset by Brendan O?Connor. Hashtags and smileys as training labels. & Hashtags and smilies works good for SA  & 6,5 \\ \hline  
    
    S6 & Murphy Choy, Michelle L.F. Cheong, Ma Nang Laik \& Koo Ping Shung & "A sentiment analysis of Singapore Presidential Election 2011 using Twitter data with census correction" & 2011 &  &  &  & "Given proper recalibration using census information, the twitter information can translate into pretty accurate information about the political landscape."  & 5,5 \\ \hline  
    
    S7 & Hassan Saif, Yulan He \& Harith Alani & Semantic Sentiment Analysis of Twitter & 2012 & Tweentor & NB & Stanford Twitter Sentiment Corpus, Health Care Reform, Obama-McCain Debate & Improvements by using semantic features on wide range topics. Acc: 83.9\%  & 9,0 \\ \hline  
    
    
    \end{longtable}
\end{sidewaystable}

\begin{sidewaystable}
    \centering
	\caption{Data extraction step, table 2/4. Showing data as per defined in the SLRP in attachment \autoref{apx:slrp}}
    \scriptsize
    \begin{longtable}{|c|p{3cm}|p{4cm}|p{0.6cm}|p{1cm}|p{1.3cm}|p{4cm}|p{3cm}|p{0.3cm}|} 
    
    \hline
    \textbf{\#ID} & \textbf{Authors} & \textbf{Title} & \textbf{Pub. Year} & \textbf{System name} & \textbf{ML Algorithm} & \textbf{Dataset} & \textbf{Findings} \& \textbf{Conclusions} & \textbf{QA} \\ 
    \hline
    
    S9 & Lei Zhang, Riddhiman Ghosh, Mohamed Dekhil, Meichun Hsu \& Bing Liu & "Combining Lexicon-based and Learning-based Methods for Twitter Sentiment Analysis" & 2011 &  & SVM, Lexical &  & Outperforms approaches that use lexical or supervised methods alone  & 8,5 \\ \hline  
    
    
    S11 & Alec Go, Lei Huang, Richa Bhayani & Twitter Sentiment Analysis & 2009 & Sentiment140 & NB, SVM & http://www.stanford.edu/~alecmgo/cs224n/ twitterdata.2009.05.25.c.zip & ~85\% - bias accuracy.  & 9,0  \\ \hline  
    
    S12 & James Spencer \& Gulden Uchyigit & Sentimentor: Sentiment Analysis of Twitter Data & 2012 & Sentimentor & NB &  & 52\% for three classes(pos,neg and objective) using bigrams without POS tagging  & 7,5 \\ \hline  
    
    S13 & Amir Asiaee T, Arindam Banerjee, Mariano Tepper \& Guillermo Sapiro  & If You are Happy and You Know It... Tweet & 2012 &  & NB,SVM,k-NN, DL &  & 82.95\% accuracy with NB on tweets about the weather  & 8,5 \\ \hline  
    
    S14 & Finn Arup Nielsen & "A new ANEW: Evaluation of a word list for sentiment analysis in microblogs" & 2011 &  & Lexical & Labeled language data created with Amazon Mechanical Turk(AMT) & The AFINN word list performs slightly better than ANEW in Twitter sentiment analysis  & 7,5 \\ \hline  
    
    S15 & Luciano Barbosa \& Junlan Feng & Robust sentiment detection on Twitter from biased and noisy data & 2010 & TwitterSA & SVM & Used Twendz, Twitter Sentiment and TweetFeel to collect data & By using data with noisy labels as input, they achieved a more abstract representation of Twitter messages than raw words.  & 8,5 \\ \hline  
    
    \end{longtable}
\end{sidewaystable}

\begin{sidewaystable}
    \centering
	\caption{Data extraction step, table 3/4. Showing data as per defined in the SLRP in attachment \autoref{apx:slrp}}
    \scriptsize
    \begin{longtable}{|c|p{2.5cm}|p{4cm}|p{0.6cm}|p{1cm}|p{1.3cm}|p{4cm}|p{3cm}|p{0.3cm}|} 
        
    \hline
    \textbf{\#ID} & \textbf{Authors} & \textbf{Title} & \textbf{Pub. Year} & \textbf{System name} & \textbf{ML Algorithm} & \textbf{Dataset} & \textbf{Findings} \& \textbf{Conclusions} & \textbf{QA} \\ 
    \hline

    
    S16 & Younggue Bae \& Hongchul Lee & "A Sentiment Analysis of Audiences on Twitter: Who Is the Positive or Negative Audience of Popular Twitterers? " & 2011 &  & Used LIWC2007 dictionary to extract sentiment & Collected tweets from celebrities and their mentions & & 5,5  \\ \hline  
    
    S17 & Mauro Cohen, Pablo Damiani, Sebastian Durandeu, Renzo Navas, Hern\'{a}n Merlino, Enrique Fern\'{a}ndez & Sentiment Analysis in Microblogging: A Practical Implementation & 2011 & - & NB & Manually gathered and annotated. 1500 tweets & Not as good accuracy as previous systems & 4,0 \\ \hline  
    
    S18 & Roberto Gonz\'{a}lez-Ib\'{a}\~{n}ez, Smaranda Muresan, Nina Wacholder & Identifying Sarcasm in Twitter: A Closer Look & 2011 & - & SMO, LogR & Data collected by using hashtag search. 900 tweets & Best result (75.89\%) was achieved in the polarity- based classification P-N. Automatic classification can be as good as human classification; however, the accuracy is still low & 9,5 \\ \hline  
    
    S19 & Akshi Kuma, Teeja Mary Sebastian & Sentiment Analysis on Twitter & 2012 & - & None. POS and own alg. &  & "The initial results show that it is a motivating technique." No stated accuracy & 7,5 \\ \hline  
    
    S20 & Alexander Pak, Patrick Paroubek & Twitter as a Corpus for Sentiment Analysis and Opinion Mining & 2010 & - & NB & http://www.stanford.edu/ ~alecmgo/cs224n/ twitterdata.2009.05.25.c.zip & ~63\% accuracy using bigrams and POS tagging & 10 \\ \hline  
    

    
    
    \end{longtable}
\end{sidewaystable}

\begin{sidewaystable}
    \centering
	\caption{Data extraction step, table 4/4. Showing data as per defined in the SLRP in attachment \autoref{apx:slrp}}
    \scriptsize
    \begin{longtable}{|c|p{3cm}|p{4cm}|p{0.6cm}|p{1cm}|p{1.3cm}|p{4cm}|p{3cm}|p{0.3cm}|} 
    
    \hline
    \textbf{\#ID} & \textbf{Authors} & \textbf{Title} & \textbf{Pub. Year} & \textbf{System name} & \textbf{ML Algorithm} & \textbf{Dataset} & \textbf{Findings} \& \textbf{Conclusions} & \textbf{QA} \\ 
    \hline
    
    S21 & Long Jiang, Mo Yu, Ming Zhou, Xiaohua Liu, Tiejun Zhao & Target-dependent Twitter Sentiment Classification & 2011 & - & SVM & "Subjectivity: Manually annotated 727 tweets for each class. Polarity: Manually annotated 268 tweets for each (pos, neg)" & "Subjectivity: 85.6\%. Polarity: 68.2\%" & 8,5 \\ \hline  
    
    S22 & Kun-Lin Liu, Wu-Jun Li \& Minyi Guo & Emoticon Smoothed Language Models for Twitter Sentiment Analysis & 2012 & ESLAM & Unigram, MLE & Sanders Corpus (5513 manually labeled tweets with one of the four different topics: Apple, Google, Microsoft, and Twitter) & ESLAM performs better than both emoteicons(distant supervised) and manually annotated tweets(fully supervised) alone & 10 \\ \hline  
    

    
    S23 & Efthymios Kouloumpis, Theresa Wilson, Johanna Moore & Twitter Sentiment Analysis: The Good the Bad and the OMG! & 2011 & - & AdaBoost.MH & Gathered by hash, http://twittersentiment.blogspot.com and iSieve Corporation for evaluating data & Hash + Emoticons result in 75\% accuracy & 8,0 \\ \hline  
    
    S25 & Apoorv Agarwal, Boyi Xie, Ilia Vovsha, Owen Rambow, Rebecca Passonneau & Sentiment Analysis of Twitter Data & 2011 & - & SVM & "Data by NextGen Invent (NGI). Manually annotated 11,875 tweets, non-bias tweets" & A gain of 4\% on unigram 3-way classification. Acc: 60.83\% & 10 \\ \hline  
    
    S26 & Asli Celikyilmaz, Dilek Hakkani-T\"{u}r \& Junlan Feng  & Probabilistic Model-based Sentiment Analysis Of Twitter Messages & 2010 &  & Used LDA to extract a polarity lexicon & Collected 2 million tweets using Twitter search API from September 2009 to June 2010. Keywords related to mobile operation. Made two manually annotated subsets from this collection. & Relatively 10\% better F-measure with unigrams than baseline for classification with text normalization and all word unigram, bigram and trigrams as features. & 7,5 \\ \hline  
    
    \end{longtable}
\end{sidewaystable}


In \autoref{tab:quality} all the individual points for the quality criteria are presented. The criteria are defined as a part of the review protocol. The average criterion score was 8.0.

\begin{table}
    \centering
    \scriptsize
    
    \setlength\tabcolsep{2pt}
    \def\arraystretch{1.4}%  1 is the default, change whatever you need
    \begin{tabular}{|c|c|c|c|c|c|c|c|c|c|c||c|} 
    
    \hline
    \textbf{\#ID} & \textbf{QC1} & \textbf{QC2} & \textbf{QC3} & \textbf{QC4} & \textbf{QC5} & \textbf{QC6} & \textbf{QC7} & \textbf{QC8} & \textbf{QC9} & \textbf{QC10} & \textbf{Total} \\
    \hline
    
    \textbf{S1} & 1.0 & 0.5 & 1.0 & 0.5 & 1.0 & 1.0 & 0.5 & 0.5 & 0.5 & 1.0 & 7.5 \\ \hline
    \textbf{S2} & 1.0 & 1.0 & 1.0 & 0.5 & 1.0 & 1.0 & 1.0 & 0.5 & 0.5 & 1.0 & 8.5 \\ \hline
    \textbf{S3} & 1.0 & 1.0 & 0.5 & 0.5 & 1.0 & 1.0 & 1.0 & 1.0 & 1.0 & 1.0 & 9.0 \\ \hline
    \textbf{S4} & 1.0 & 1.0 & 0.5 & 0.5 & 1.0 & 1.0 & 1.0 & 0.5 & 0.5 & 1.0 & 8.0 \\ \hline
    \textbf{S5} & 1.0 & 0.5 & 0.5 & 0.5 & 1.0 & 1.0 & 0.0 & 0.5 & 1.0 & 0.5 & 6.5 \\ \hline
    \textbf{S6} & 1.0 & 0.5 & 0.5 & 0.5 & 0.0 & 0.5 & 0.0 & 0.5 & 1.0 & 1.0 & 5.5 \\ \hline
    \textbf{S7} & 1.0 & 1.0 & 1.0 & 0.5 & 0.5 & 1.0 & 1.0 & 1.0 & 1.0 & 1.0 & 9.0 \\ \hline
    \textbf{S8} & 1.0 & 1.0 & 0.5 & 1.0 & 1.0 & 1.0 & 1.0 & 0.5 & 0.5 & 1.0 & 8.5 \\ \hline
    \textbf{S9} & 1.0 & 1.0 & 1.0 & 0.5 & 1.0 & 1.0 & 1.0 & 1.0 & 0.5 & 1.0 & 9.0 \\ \hline
    \textbf{S10} & 1.0 & 1.0 & 0.5 & 1.0 & 1.0 & 1.0 & 0.5 & 0.5 & 0.5 & 0.5 & 7.5 \\ \hline
    \textbf{S11} & 1.0 & 0.5 & 1.0 & 0.0 & 1.0 & 1.0 & 1.0 & 1.0 & 1.0 & 1.0 & 8.5 \\ \hline
    \textbf{S12} & 1.0 & 1.0 & 0.0 & 1.0 & 0.5 & 1.0 & 0.0 & 1.0 & 1.0 & 1.0 & 7.5 \\ \hline
    \textbf{S13} & 1.0 & 1.0 & 1.0 & 1.0 & 0.5 & 0.5 & 0.5 & 1.0 & 1.0 & 1.0 & 8.5 \\ \hline
    \textbf{S14} & 0.5 & 0.5 & 0.5 & 1.0 & 0.0 & 0.5 & 0.5 & 0.5 & 0.5 & 1.0 & 5.5 \\ \hline
    \textbf{S15} & 1.0 & 0.0 & 0.0 & 0.5 & 1.0 & 1.0 & 0.0 & 0.0 & 0.0 & 0.5 & 4.0 \\ \hline
    \textbf{S16} & 1.0 & 1.0 & 0.5 & 1.0 & 1.0 & 1.0 & 1.0 & 1.0 & 1.0 & 1.0 & 9.5 \\ \hline
    \textbf{S17} & 1.0 & 1.0 & 1.0 & 1.0 & 1.0 & 1.0 & 0.5 & 0.5 & 0.0 & 0.5 & 7.5 \\ \hline
    \textbf{S18} & 1.0 & 1.0 & 1.0 & 1.0 & 1.0 & 1.0 & 1.0 & 1.0 & 1.0 & 1.0 & 10,0 \\ \hline
    \textbf{S19} & 1.0 & 1.0 & 0.5 & 0.5 & 1.0 & 1.0 & 0.5 & 1.0 & 1.0 & 1.0 & 8.5 \\ \hline
    \textbf{S20} & 1.0 & 1.0 & 1.0 & 1.0 & 1.0 & 1.0 & 1.0 & 1.0 & 1.0 & 1.0 & 10,0 \\ \hline
    \textbf{S21} & 1.0 & 1.0 & 1.0 & 0.5 & 1.0 & 1.0 & 0.5 & 0.5 & 0.5 & 1.0 & 8.0 \\ \hline
    \textbf{S22} & 1.0 & 1.0 & 1.0 & 1.0 & 1.0 & 1.0 & 1.0 & 1.0 & 1.0 & 1.0 & 10,0 \\ \hline
    \textbf{S23} & 0.5 & 1.0 & 1.0 & 0.5 & 0.5 & 0.5 & 0.5 & 1.0 & 1.0 & 1.0 & 7.5 \\ \hline
     &  &  &  &  &  &  &  &  &  & \textbf{Avg} & \textbf{8.0} \\ \hline
    
    \end{tabular}
    
    \caption{Quality Assessment for the studies. Each QC can give 0, 0.5 or 1 point. The average score is 8.0}
    \label{tab:quality}
\end{table}

\subsection{Twitter Sentiment Analysis: State-of-the-Art}
\label{sec:stateofart}
This chapter presents some of the state-of-the-art Twitter Sentiment Analysis(TSA) approaches, and the techniques that are used. The vocabulary used on Twitter makes it hard for traditional natural language processing systems to understand, because they are usually trained on a formal language. This has made researchers exploit some of the special features that the web language, and Twitter gives us, e.g abbreviations and emoticons.
	
		\subsubsection{Data Collection and Preprocessing}
		Most of the data used in TSA research is collected through the Twitter API, either by searching for a certain topic/keyword or by streaming real-time data. Some datasets from related research on the Twitter platform has also been made available for other research projects, as an alternative to collecting a complete data set from scratch. Some approaches specialize on certain domains, while others query for tweets containing emoticons (\emph{':)', ':)'}) to train a cross-domain classifier~\citep{article:go}. The idea behind the emoticon approach is to make sure that the collected tweets contain subjectivity, but these training sets alone are limited to binary classification only (positive/negative classification).

After the data has been collected it should go through a filtering process. First, all non-English tweets are removed, then the Twitter specific symbols and functions described in \autoref{tab:features} would normally be filtered out. As mentioned, a study by \cite{article:go} used ':)' and ':(' as a label for the polarity in their training data, and thus they did not remove these emoticons, but the URLs and usernames were replaced by a nomial ('URL' or 'USERNAME'). They also removed the query term from the text so that it would not affect the classification.

\cite{article:omg} used a hashtagged data set (HASH) in addition to an emoticon data set (EMOT) from http://sentiment140.com. The hashtagged set is a subset of the Edinburgh Twitter corpus which consists of 97 million tweets~\citep{article:edinburgh}.

\begin{table}[]
\centering
\begin{tabular}{|l|l|p{8cm}|}
\hline
RT & Retweet & Reposting another user’s tweet \\ \hline
@ & Mention & Tag used to mention another user \\ \hline
\# & Hashtag & Hashtags are used to tag a tweet to a certain topic. Have become popular recently, and is also used on other platforms \\ \hline
:),:-),$\wedge\wedge$ & Emoticon & Hashtags are used to tag a tweet to a certain topic. Have become popular recently, and is also used on other platforms \\ \hline
URL & URL & Typically a link to an external resource, e.g a new article or a photo \\ \hline
\end{tabular}
\caption{Features that are usually removed from the tweets.}
\label{tab:features}
\end{table}

Some approaches have also experimented with normalizing the tweets, and removing redundant letters, e.g ''loooove'' and ''crazyyy'', that are often used in tweets. Redundant letters like these are sometimes used to express a stronger sentiment, and it has therefore been experimented with trimming down to one additional redundant letter('loooove' = 'loove' instead of love), so that the stronger sentiment can be taken into consideration by a score/weight adjustment for that feature.

\subsubsection*{Part-of-speech tagging}
Part-of-Speech (\nom{POS}{Part-of-Speech}) tagging is a well-known process for marking the words of a sentence. Adjectives, adverbs and personal pronouns have shown good indicators for subjectivity, which has made POS tagging a good technique for filtering out objective tweets before the polarity classification. Early research on TSA showed that the challenging vocabulary made it harder to tag the tweets with a good accuracy; however, in 2010 \cite{article:gimpel} made a POS tagger that aimed at marking tweets. It performed very well in their experiments (almost 90\% accuracy).

		
		\subsubsection{Subjectivity Classification}
		The most used strategy for TSA is a two-step strategy where the first step is subjectivity classification and the second step is the polarity classification. The goal for the subjectivity classification is to separate subjective and objective tweets.

One of the most used techniques for this task is POS tagging. \cite{article:pak} found several indicators of subjectivity by counting word frequencies in a subjective set versus an objective set. They found that interjection and personal pronouns were the strongest indicators of subjectivity in their set. In their paper,~\cite{article:pak} concluded that utterances were a strong indicator of subjectivity, but referring to the tree tag UH. According to the POS guidelines~(\cite{treebank}), tree tag UH is, how ever, not utterances, but rather interjections. In~\cite{article:jiang} used normalization, POS tagging, word stemming and syntactic parsing for the subjectivity classification task. The idea was that normalization of features would give better recall.

Previous research has also explored the use of noisy data and distant supervised methods such as emoticons and hashtags for the subjectivity classification, where any match from a given lexicon will classify the tweet as subjective.

		
		\subsubsection{Polarity Classification}
		The final part of the analysis is the polarity classification (positive/negative). While TSA is not yet considered mature, SA for longer texts, i.e documents and reviews, has been explored for years~\citep{book:pang}. Different techniques and algorithms that have proven worthy for longer texts has also been applied to sentence level SA with various success. Among these techniques, supervised learning methods like naive Bayes classifier (NB), maximum entropy (MaxEnt) and support vector machines (SVM) are the most used. The limited amount of attributes in tweets makes the feature vectors shorter than in documents. For that reason there are no guarantee that algorithms that perform well on document level SA will be the best alternatives for classifying short texts like tweets.

Some approaches have also experimented with a combination of lexicon-based methods and  machine learning~\citep{article:mudinas}. They perform an entity-level sentiment analysis as the first step. Then they use tweets that are likely to be opinionated in a lexicon-based method. The last step of their process is to train a classifier to assign the sentiment value. This approach makes it possible to train the classifier without manually labeling the data, as they’re using the data from the lexicon-based method.\vspace{8mm}

\noindent
\textbf{Supervised learning} \\
\noindent
methods require some sort of training data to create an inferred function for classification tasks. These data would preferably be manually annotated texts, but as this can be a labor-intensive task, some research has experimented with emoticons or a collection of hashtags as labels for positive/negative tweets. This is done by making assumptions, such as all tweets containing positive emoticons are positive, and that all who contain negative emoticon’s are negative.

Among the machine learning algorithms that perform well on TSA are NB, SVN and MaxEnt. While SVN normally beats NB and MaxEnt on longer texts, it seem to have some trouble with outperforming the NB when feature vectors are shorter, e.i shorter texts. \cite{article:bermingham} has shown this in their comparison of SVN and NB for microblogs.\vspace{8 mm}

\noindent
\textbf{Unsupervised learning} \\
\noindent Of the unsupervised methods in TSA, the lexicon-based seem to be the most used approach. This technique requires a lexicon with a sentiment score for each word. When using such lexicons the classifier can look up all the words in the feature vector, e.g a bag of words, and check the sentiment score if the feature exists in the lexicon. Hence it will not need any training beforehand.
	
Popular sentiment lexicons are SentiWordNet and General Inquirer. Some have also made custom extensions of these lexicons that included manually annotated emoticons and hashtags as well as words. \cite{article:afinn} made a sentiment lexicon called AFINN, specialized for Twitter. It contains a lot of words from the vocabulary used in social networks. AFINN supports slang and abbreviations, e.g ‘n00b’, ‘lol’ and ‘wtf’. This lexicon was made as a response to the ANEW lexicon which works better for document level SA since it does not support the Twitter language.