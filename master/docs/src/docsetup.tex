\documentclass[12pt,a4,twoside,openleft]{book}		
%Book style with A4 paper, chapeters start on left page only, using times fonts

\usepackage[T1]{fontenc}
\usepackage{ae,aecompl}
\usepackage{times}
% Selects font encoding

\usepackage{url}
% breakes long urls prettier over more than one line

\usepackage[colorlinks=true,pageanchor=true,linkcolor=black,anchorcolor=black,filecolor=black,citecolor=black,menucolor=black,pagecolor=black,urlcolor=black,bookmarksopen=true,bookmarksopenlevel=1]{hyperref}
% Creates hyperlinks of references/urls. Default color is blue but the settings above change the colors to black.

\usepackage{parskip}
% Starts new paragraphs without indentation but with some space between the new and the previous paragraph.

\usepackage{multirow}
% Used to create tables with rows/cols spanning over se

\usepackage{graphicx}
% Used to include images with the includegraphics command

\usepackage{pdfpages}
% Used to import pages from or whole pdf-documents




%\usepackage{makeidx}
%\makeindex
% Needed to create glossaries


% \usepackage[number=none]{glossary}
% \makeglossary
% To create glossary and list of acronyms. Other packages may also be used. Page numbering is turned off in the final list.

% Creates a new glossary for abbreviations
% \newglossarytype[abr]{abbr}{abt}{abl}

% \newglossarytype[alg]{acronyms}{acr}{acn}
%\newcommand{\abbrname}{Abbreviations} 
%\newcommand{\shortabbrname}{Abbreviations}
%\makeabbr




% The below code sets the margins of the document.
% See: http://web.image.ufl.edu/help/latex/margins.shtml for an explanation of the margins.

\oddsidemargin 5mm
\evensidemargin 5mm
\textwidth 150mm
\topmargin 0mm
\headheight 0mm
\textheight 225mm
%\footheight 0mm






\usepackage{color}
% To give color to the text. It is only used to highlight the instructions in this document and it can be removed.