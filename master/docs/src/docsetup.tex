\documentclass[12pt,a4paper]{book}		
%Book style with A4 paper, chapeters start on left page only, using times fonts

\usepackage[T1]{fontenc}
\usepackage{ae,aecompl}
\usepackage{times}
% Selects font encoding
\usepackage{rotating}

\usepackage{subcaption}

\usepackage{caption}
\usepackage{float}

\usepackage{geometry}
\geometry{b5paper}

\usepackage{pgfplots}

\usepackage{pdfpages}

\usepackage{longtable}
\usepackage{supertabular}
\usepackage{cleveref}
\usepackage{lscape}

\usepackage{array}
\usepackage{calc}

\usepackage{natbib}								% Correct citations


\usepackage{url}
% breakes long urls prettier over more than one line

\usepackage[colorlinks=true,pageanchor=true,linkcolor=black,anchorcolor=black,filecolor=black,citecolor=black,menucolor=black,pagecolor=black,urlcolor=black,bookmarksopen=true,bookmarksopenlevel=1]{hyperref}
% Creates hyperlinks of references/urls. Default color is blue but the settings above change the colors to black.

\usepackage{parskip}
% Starts new paragraphs without indentation but with some space between the new and the previous paragraph.

\usepackage{multirow}
% Used to create tables with rows/cols spanning over se

\usepackage{graphicx}
% Used to include images with the includegraphics command

\usepackage{pdfpages}
% Used to import pages from or whole pdf-documents

\usepackage{amsmath}
% Used for math symbols
\usepackage{amssymb}


% Used for abbriviations
\usepackage{nomencl}
\renewcommand{\nomname}{Abbreviations}
\renewcommand{\nomlabel}[1]{\textbf{#1}} % make abbreviations bold
\makenomenclature
\newcommand*{\nom}[2]{#1\nomenclature{#1}{#2}}
\cleardoublepage

%\usepackage{makeidx}
%\makeindex
% Needed to create glossaries


% \usepackage[number=none]{glossary}
% \makeglossary
% To create glossary and list of acronyms. Other packages may also be used. Page numbering is turned off in the final list.

% Creates a new glossary for abbreviations
% \newglossarytype[abr]{abbr}{abt}{abl}

% \newglossarytype[alg]{acronyms}{acr}{acn}
%\newcommand{\abbrname}{Abbreviations} 
%\newcommand{\shortabbrname}{Abbreviations}
%\makeabbr

% Creates a new command for argmax with underset
\newcommand{\argmax}[1]{\underset{#1}{\operatorname{arg}\,\operatorname{max}}\;}




% The below code sets the margins of the document.
% See: http://web.image.ufl.edu/help/latex/margins.shtml for an explanation of the margins.

%\oddsidemargin 5mm
%\evensidemargin 5mm
%\textwidth 150mm
%\topmargin 0mm
%\headheight 0mm
%\textheight 225mm
%\footheight 0mm



% Hyphens
\hyphenation{Sem-Eval Senti-Map Senti-Graph Senti-Stack Java-Script hand-ling domene-uavhengig Timeline-Stats Tweet-Count}

% Contents 
\setcounter{secnumdepth}{3}
% \setcounter{tocdepth}{3}

\usepackage{color}
% To give color to the text. It is only used to highlight the instructions in this document and it can be removed.