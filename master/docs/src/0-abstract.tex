\section*{Abstract}
\addcontentsline{toc}{chapter}{Abstract}

\textcolor{blue}{10-15 lines on motivation, 10-15 lines on approach, 3-5 research questions (named RQ1, . ..), 3-5 main 
contributions (named C1, ...) being briefly coupled to the RQi. Start with the abstract ASAP! Keep the abstract to 1 
page. \textbf{Example from Bj{\o}rnson's thesis below} (with smaller font to fit in page):}

\small

\textcolor{green}{Reports of software a development projects that miss schedule, exceeds budget and deliver products 
with poor quality are abundant in the literature. Both researchers and the industry are seeking methods to counter 
these trends and improve software quality.}

\textcolor{green}{Software Process Improvement is a systematic approach to improve the capabilities and performance of 
software organizations. One basic idea is to assess the organizations� current practice and improve their software 
process on the basis of the competencies and experiences of the practitioners working in the organization. A major 
challenge is to create strategies and mechanisms for managing relevant and updated knowledge about software development 
and maintenance. Insights from the field of knowledge management are therefore potentially useful in software process 
improvement efforts to facilitate the creation, modification, and sharing of software processes in any organization.}

\textcolor{green}{In the work presented in this thesis, we have made an overview of empirical studies on the effect of 
knowledge management in software engineering. We have categorized these studies according to a framework and we report 
findings on the major concepts that have been investigated empirically, as well as the research methods applied within 
the field. We have also followed software process improvement initiatives in three companies through action research 
studies. We examined socialization through a mentor program, and codification of software process through two 
approaches, one based on the Rational Unified process and one using Process Workshops. Finally we have suggested a 
revised method for project reviews, which we have shown empirically in a controlled experiment to be more effective 
than previously suggested methods for our chosen context.}

\textcolor{green}{We have classified the work in this thesis within three main themes:}
\vspace{-0.5cm}
\begin{list}{$\bullet$}{}
 \setlength{\parskip}{0pt}
 \item [\textcolor{green}{RT1}] \textcolor{green}{Overview of previous research on knowledge management in software 
 engineering.}
 \item [\textcolor{green}{RT2}] \textcolor{green}{Application of knowledge management to improve the software process 
 through codification of knowledge.}
 \item [\textcolor{green}{RT3}] \textcolor{green}{Application of knowledge management to improve the software process 
 through sharing of knowledge from person to person}
\end{list}
\vspace{-0.5cm}

\textcolor{green}{The main contributions are:}
\vspace{-0.5cm}
\begin{list}{$\bullet$}{}
 \setlength{\parskip}{0pt}
 \item [\textcolor{green}{C1}] \textcolor{green}{An overview of the research literature on empirical studies of 
 knowledge management in software engineering.}
 \item [\textcolor{green}{C2}] \textcolor{green}{A method for tailoring the Rational Unified Process to the development 
 process of a software consulting company.}
 \item [\textcolor{green}{C3}] \textcolor{green}{Improvements of the Process Workshops method by contextualization.}
 \item [\textcolor{green}{C4}] \textcolor{green}{Improvement of the root-cause analysis phase of the lightweight Post 
 Mortem Analysis for more effective project reviews.}
 \item [\textcolor{green}{C5}] \textcolor{green}{Proposed methods to increase the learning effect of mentor programs in 
 software engineering.}
\end{list}

\normalsize
\cleardoublepage