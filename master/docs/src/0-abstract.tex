\section*{Abstract}
\addcontentsline{toc}{chapter}{Abstract}

The social micro-blog site Twitter grows in user base each day and has become an attractive platform for companies, politicians, marketeers, and others wishing to share information and/or opinions. With a growing user market for Twitter, more and more systems and research are released for taking advantage of its informal nature and doing opinion mining and sentiment analysis. 

This master thesis describes a system for doing Sentiment Analysis on Twitter data and experiments with grid searches on various combinations of machine learning algorithms, features and preprocessing methods to achieve so. The classification system is fairly domain independent and performs better than baseline. 

This system is designed to be fast to be able to classify big amounts of data and tweets in a stream, and exposes an application program interface (API) to easily provide data to applications or end users. 

Three visualisation applications are implemented, showing how to use the API and examples of what sentiment data can be used as.

The main contributions are: 

\begin{itemize}
\item[\textbf{C1}] This thesis provides a definition of the state-of-the-art for Twitter Sentiment Analysis.

\item[\textbf{C2}] A general system architecture for doing Twitter Sentiment Analysis is implemented. 

\item[\textbf{C3}] Different machine learning algorithms are compared for the task of identifying sentiments in short messages in a fairly semi-independent domain.

\item[\textbf{C4}] Visualisation applications are implemented, showing how to use data from the generic system and examples of how to show sentiment analysis data.
\end{itemize}

\clearpage

\section*{Sammendrag}

Den sosiale mikrobloggingsnettstedet Twitter er i kontinuerlig vekst og har blitt et attraktivt verkt\o y for bedrifter, politikere, markedsf\o rere og andre som \o nsker \aa~dele informasjon og/eller meninger. Med \o kende brukermasse, kommer det ut flere og flere systemer og forskningsresultater relatert til Twitter, sentimentanalyse og meningsm\aa linger.

Denne masteroppgaven definerer et system for \aa~utf\o re sentimentanalyse p\aa~data fra Twitter og oppn\aa r dette ved \aa~eksperimentere med parameters\o k p\aa~forskjellige kombinasjoner av maskinl\ae ringsalgoritmer, funksjoner og preprosesseringsmetoder. Det utviklede systemet er forholdsvis domene-uavhengig og har bedre ytelse enn baseline.

Dette systemet ble designet for \aa~takle stor mengde data og klassifisere tweets i datastr\o mmer. For enkelt \aa~tilby data til sluttbrukeren, ble et "application program interface"~(API) definert.

Tre visualiseringsapplikasjoner ble implementert for \aa~vise bruken av API og hvordan sentiment data kan bli brukt og vist.

Hovedbidragene fra oppgaven er f\o lgende:

\begin{itemize}
\item[\textbf{C1}] Oppgaven definerer en state-of-the-art for Twitter SentimentAnalyse (TSA).

\item[\textbf{C2}] Et generelt systemarkitektur for TSA ble utviklet og implementert.

\item[\textbf{C3}] Forskjellige maskinl\ae ringsalgoritmer er satt i mot hverandre for \aa~ best finne sentiment i korte meldinger p\aa~en relativt domeneuavhengig m\aa te.

\item[\textbf{C4}] Visualiseringsapplikasjoner er implementert, og viser hvordan en kan bruke dataen fra det generelle systemet og hvordan sentimenter kan bli presentert og brukt.
\end{itemize}

\clearpage