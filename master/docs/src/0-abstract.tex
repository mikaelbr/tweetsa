\section*{Abstract}
\addcontentsline{toc}{chapter}{Abstract}

The social micro-blog site Twitter grows in user base each day and has become an attractive platform for companies, politicians, marketeers, and others wishing to share information and/or opinions. With a growing user market for Twitter, more and more systems and research are released for taking advantage of its informal nature and doing opinion mining and sentiment analysis. 

This master thesis describes a system for doing Sentiment Analysis on Twitter data and experiments with grid searches on various combinations of machine learning algorithms, features and pre-processing methods to achieve so. The classification system is fairly domain independent and performs better than base line. 

This system is designed to be fast to be able to classify big amounts of data and tweets in a stream, and exposes an application program interface (API) to easily provide data to applications or end users. 

Two visualisation applications are implemented, showing how to use the API and examples of what sentiment data can be used as.

The main contributions are: 

\begin{itemize}
\item[\textbf{C1}] This thesis provides a definition of the state-of-the-art for Twitter Sentiment Analysis.

\item[\textbf{C2}] A general system architecture for doing Twitter Sentiment Analysis is implemented. 

\item[\textbf{C3}] Different machine learning algorithms are compared for the task of identifying sentiments in short messages in a fairly semi-independent domain.

\item[\textbf{C4}] Visualisation applications are implemented, showing how to use data from the generic system and examples of how to show sentiment analysis data.
\end{itemize}

\cleardoublepage