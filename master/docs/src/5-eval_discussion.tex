\chapter{Evaluation and Discussion of Results}


\textcolor{blue}{(8-10 pages) In this chapter you assess your results. Identify your contributions. Possible theory 
building (establish cause-effect). Compare to other work described in chapter 2. Suggestions for improvements. Discuss 
construct-, internal-, external- and conclusion-validity. The major challenge in this chapter is usually which axis you 
want to structure your discussion around: research questions, contributions or studies. Find what works best for you 
and your studies.}

\textcolor{green}{Evaluation of research questions}

\textcolor{blue}{If you did not answer these questions in the results chapter, now is the time to revisit.}

\textcolor{green}{Evaluation of Contributions}

\textcolor{blue}{How does our contributions fit with the state of the art we described in chapter 2? Do they extend the 
field? In what way? How do your contributions compare to your research questions? Do you have your own reflections on 
the contributions.}

\textcolor{green}{Evaluation of Validity Threats}

\textcolor{blue}{What are the major threats to our research? Mention the major threats like:}

\begin{list}{$\bullet$}{}
  \item Internal Validity 
  \item External Validity
  \item Construct Validity
  \item Conclusion Validity
\end{list}

\textcolor{blue}{Note that you might have to discuss these separately for each study, and every validity might not be applicable depending on what research method you have used.}

\textcolor{green}{Reflections on the research context}

\textcolor{blue}{Optional. But it is often good to reflect on the (project) context of your research and how it has 
affected you and your research.}