The most used strategy for TSA is a two-step strategy where the first step is subjectivity classification and the second step is the polarity classification. The goal for the subjectivity classification is to separate subjective and objective tweets.

One of the most used techniques for this task is POS tagging. \cite{article:pak} found several indicators of subjectivity by counting word frequencies in a subjective set vs an objective set. They found that utterances and personal pronouns were the strongest indicators of subjectivity in their set. \cite{article:jiang} used normalization, POS tagging, word stemming and syntactic parsing for the subjectivity classification task. The idea was that normalization of features would give better recall.

Previous research has also explored the use of noisy data and distant supervised methods such as emoticons and hashtags for the subjectivity classification, where any match from a given lexicon will classify the tweet as subjective.