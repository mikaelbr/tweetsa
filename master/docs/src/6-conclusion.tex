\chapter{Conclusion}


\textcolor{blue}{(2-3 pages) Time for the conclusion, be short and try to nail down the essence. This should usually 
list the major conclusions from your previous discussion. There should also be a section on possible future directions 
for your work in this chapter.}


\section{Contribution}

In this project we have defined the state of the art for Twitter Sentiment Analysis systems, and implemented a generic system with an architecture capable of classifying many tweets per second in a live stream. We found that, by extending the Twitter API and just attaching a value for sentiments, the data can easily be used to implement visualisation applications, or extending existing Twitter based systems. 

We have tested different machine learning algorithms, reported to work well with TSA, and found that SVM and MaxEnt performs well on short messages in a semi-independent domain. 

\section{Future Work}

A natural way to extend this work, is to add additional classification algorithms as models. There are also several different features that could be investigated and more combinations of pre-processing methods. 

To improve the classification on tweets, and thus make the system less domain independent, more Twitter specific features can be used. E.g. lexica developed specifically for social media, like the~\citet{article:afinn} lexicon. 

It could also be interesting adding POS-tagging as an experiment, using it to classify subjectivity as shown to work by~\citet{article:pak}.

While developing the visualisation application SentiMap, another more specific feature could be seen in the tweets streamed by Twitter. As the tweets were restricted by geographical location, most of them originated from hand held devices. Apples iPhone, has their own smilies, originally intended for the Japan user marked, called Emoji\footnote{\url{http://en.wikipedia.org/wiki/Emoji}}. Emoji smilies are different smilies and icons representing situations of sentiments, but not constructed from ASCII characters like regular smilies (e.g. $:($ and $:)$). By translating these Emojies to sentiments like $||Happy||$ or $||Sad||$ they could be used as features.