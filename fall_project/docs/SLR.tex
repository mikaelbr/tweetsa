This section will describe what a systematic literature review (SLR) is and how it is conducted. 

A SLR is a methodological and formal way of retrieving information and doing literature search regarding a topic. SLR has been used to a great extent in fields like medicine, but not as much in computer science. An SLR uses several steps to accomplish the literature search and uses a systematic literature review protocol to document these steps. This way the literature search can be reproducible and is documented~\citep{paper:slrdesc}.

\subsection{Performing a SLR}

An SLR consists of three phases with 13 steps in total: planning, conducting and reporting, as defined by \cite{paper:slrdesc} and \cite{master:slr}:

\subsubsection{Planning}

\begin{description}

	\item[1. Identification of the need for a review] \hfill \\
		Identify whether or not the researchers need to review literature using SLR.

	\item[2. Commissioning a review] \hfill \\
		Organizations need to have the resources to do an SLR and commission the review.

	\item[3. Specifying the research question(s)] \hfill \\
		Define research questions (RQ) to specify what the researchers hope to find information about. The RQs are used as basis for, among others, the search, inclusion, and data collection in the conduction phase. 

	\item[4. Developing a review protocol] \hfill \\
		Develop a protocol providing documentation for all steps for conducting an SLR.
	

	\item[5. Evaluating the review protocol] \hfill \\
		Have an independent party or person evaluate the protocol.

\end{description}

% This probably shouldn't be here but in the SLR chapter..
% In this report both step 1 and 2 are assumed completed.

\subsubsection{Conducting}

\begin{description}

	\item[1. Identification of research] \hfill \\
		Locate as many papers as possible relevant to the previously defined and documented RQs. The search strategy needs to be documented with clear definitions for search domains and search string.

	\item[2. Selection of primary studies] \hfill \\
		Use a set of inclusion criteria (IC) to select what papers to use for the review. This can be done in two steps, by filtering on only title and abstract or based on the full text paper. 

	\item[3. Study quality assessment] \hfill \\
		Use a set of quality criteria (QC) to further filter down the paper collection. Use the QCs to set a score of each paper. Set a lower score limit to filter irrelevant or papers of poor quality.

	\item[4. Data extraction and monitoring] \hfill \\
		Extract data from all the papers selected and included. This data collection should be documented by the protocol, defining what features and information to extract.
	

	\item[5. Data synthesis] \hfill \\
		Synthesise the data to be able to answer the RQs. 
\end{description}

% This probably shouldn't be here but in the SLR chapter..
% For this report, the data synthesis is included as a part of the data extraction. 

\subsubsection{Reporting}


\begin{description}

	\item[1. Specifying dissemination strategy] \hfill \\
		Specify how the review result shall be presented. 

	\item[2. Formatting the main report] \hfill \\
		Report the literature review in its own report or as a part of a report.

	\item[3. Evaluating the report] \hfill \\
		Have an independent expert in the field evaluate the SLR. 

\end{description}

