
For this project we introduced three main goals:
\begin{itemize}
\item Establish the state-of-the-art for sentiment analysis systems on Twitter
\item Developed a way to distribute sentiment classifications.
\item Develop an architecture for a sentiment analysis system by implementing a basic system.
\end{itemize}

In the first section of this chapter we will discuss whether we reached our first goal. In the second section we will discuss more about the second and third goals. 

In the last section we will discuss our experiences with conducting a systematic literature review. 


\subsection{Establishing the state-of-the-art}
From the results of our systematic literature review, we have seen that a lot of experiments have been done in the field of Twitter Sentiment Analysis over the last years. We had little trouble finding enough studies to establish the state-of-the-art. To limit the number of papers, the articles from before 2008 were filtered out.

Establishing the performance of state-of-the-art TSA systems was not a simple task. The different experiments have used different data and test sets, and some have used domain specific data while others have trained their learners for cross-domain classification. Another problem that has to be faced is the time consuming task of manually labeling tweets for training and testing. As a solution to this, researchers have experimented with automatic labeling data sets with the usage of noisy data that can lead to a training set that is biased towards these noisy features. This will in turn affect the accuracy of the system, and it will not be a realistic measure. 

\subsection{Distributing sentiments and system architecture}

When developing a way of distributing the sentiment data, we wanted a way that was natural for developers so that they would easily be able to understand our API and start using it straight away. API development is a demanding process, and many teams devote a lot of time to finding the best way of doing it. Instead of trying to define our own interface from scratch, we decided that reflecting the existing Twitter API was the best way to go. By doing this, we also have all of the documentation of the API predefined and would not have to use time and resources of doing this ourselves. 

The only problem we had with extending the Twitter API was the OAuth protocol. This protocol has a lot of documentation and it can be quite advanced. We first tried to make our API Layer just pass on the authentication data (API key and secret) to Twitter API, without us meddling with it. We found out that we would have to implement an OAuth server ourselves and send the key and secret on to Twitter through that server. This was a task we did not have the time to execute. 

The API Layer is designed to work asynchronously and be very scalable. In theory it should also be that. We have not done as much testing as we have liked to do. If we had had more time we would have stress-tested the system with more simultaneously requesting clients. Both for the API Layer and the classification server. 

By using only asynchronously requests and message sending, all communication between the API Layer and the Sentiment classification server happens in parallel. This means that even though there are 20 tweets lined up for classification, the total time will seem be only somewhat higher than the time for classifying one. The CPU will not be able to handle more parallel operations than it has cores. So while it seems like the processing is carried out in parallel, it is not entirely so. Given enough tasks the CPU will queue up the processes. This is still a lot better than doing all classifications sequentially. 

We designed an architecture for doing sentiment classification based on other existing systems. We feel that the architecture is robust and as modular as we want it to be. However, since we have not implemented an advanced sentiment classifier yet, we are unsure of whether or not the architecture will hold if, scaling up the system. There might be, if done further work on the sentiment analysis system, some changes should be made to the overall architecture. 


\subsection{Structured Literature Review}

By performing an SLR we found that it was a good process for documenting our literature search and had some very good techniques for deciding what papers to include and build our work on. We also found that carrying out a structured literature review is a very time consuming process. There are few examples of how to do a structured literature review in the computer science field, which, at first, made it hard to get a working knowledge of how to perform one. In a perfect situation we would re-do some of our processes after gaining more knowledge of how to conduct a review. This is something we did not have the time to do. 

We feel that we have used the SLR process to such a degree that we have gained sufficient knowledge to define the state-of-the-art for a sentiment analysis system on Twitter data. 