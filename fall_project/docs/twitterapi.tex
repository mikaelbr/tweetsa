Twitter allows developers and others to access their data by the means of an application program interface (API). The API is implemented by using Representational State Transfer (REST). REST is a style of software architecture for distributing data across the World Wide Web (WWW). By using the protocol HTTP's vocabulary of methods, developers can get, insert, delete and update data on Twitter, given the proper access. Some of the important methods are GET for retrieving data, POST for inserting, updating and sending data, DELETE for removing data.~\citep{article:rest}

Twitter uses OAuth for authentication. OAuth provides a way for clients to access resources on behalf of end users or other clients. OAuth is also used to provide third party applications access to a users data on a service, without sharing the users user name or password.~\citep{site:oauth}

After migrating to version 1.1 of their API, Twitter now has a limit on the number of requests an end user can do. As end users authenticate using OAuth, Twitter can identify them and limit their access if overused. The limitations are divided into 15 minute intervals. Some services on the API are limited to 15 requests per time window, other services are limited by 180 requests. E.g. the search service is limited by 180 requests, but the service for retrieving tweets from a Twitter list is limited by 15.~\citep{site:twitterlimit}

Almost any aspect of Twitter is covered by the REST API. The most interesting parts are the ones to retrieve tweets in various ways. There are methods for retrieving entire user time-lines, mentions of a user, favourite tweets for a user, tweets based on geographical annotated locations, and so on.~\citep{site:twitterapi}

There are also real time streaming services in the API. Streams are not based on the Twitter REST API. For streaming a persistent HTTP connection is opened to the API. This way a client would not have to continuously poll the REST API to register changes or new data. A client would be provided by real time data from Twitter, but only have one single connection opened.~\citep{site:twitterstream}

The Twitter API uses JavaScript Object Notation (JSON) as their format for response. JSON is a lightweight data format often used as an alternative to XML. JSON was created by~\citeauthor{site:json} to be a subset of the JavaScript Programming Language but still be language-independent. JSON has a simple structure and aims to be minimal, portable and textual and has support for four primitive types (strings, numbers, booleans, and null) and two structure types (array and objects).~\citep{site:json}

