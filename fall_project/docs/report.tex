\documentclass[a4paper, 11pt]{book}
\usepackage[T1]{fontenc}
\usepackage[utf8]{inputenc}
\usepackage[english]{babel}
\usepackage{graphicx} % support graphics
\usepackage{hyperref} % links in the document
% \usepackage{fullpage} % smaller margins
\usepackage{float} % position of figures
\usepackage{color}
\usepackage{calc}
\usepackage{xargs}
\usepackage[titletoc]{appendix}
\usepackage{natbib}								% Correct citations
\usepackage{rotating}
\usepackage{longtable}
\usepackage{supertabular}
\usepackage{cleveref}
\usepackage{lscape}

\usepackage{array}

% Use a new command to wrap std functionality
\newcommand{\insertfig}[3][0.7]{
	\begin{figure}[ht]
	    \begin{center}
	        \includegraphics[width=#1\textwidth]{./figs/#2}
	    \end{center}
	    \caption{#3}
	    \label{fig:#2}
	\end{figure}
}


\makeatletter
\let\mcnewpage\newpage
\newcommand{\changenewpage}{%
  \renewcommand\newpage{%
    \if@firstcolumn
      \hrule width\linewidth height0pt
      \columnbreak
    \else
      \mcnewpage
    \fi
}}
\makeatother

% Author
% Fill in here, and use commands in the text. 
\newcommand{\thesisAuthor}{Mikael Brevik \& Øyvind Selmer}
\newcommand{\thesisTitle}{Twitter Sentiment Analysis: \\ State-of-the-Art and an \\ Architectural Implementation}
\newcommand{\thesisType}{Specialization project}
\newcommand{\thesisDate}{fall 2012}


% PDF info
\hypersetup{pdfauthor={\thesisAuthor}}
\hypersetup{pdftitle={\thesisTitle}}
\hypersetup{pdfsubject={\thesisType}}

% Configure different colors
\definecolor{darkred}{rgb}{0.5,0,0}
\definecolor{darkgreen}{rgb}{0,0.5,0}
\definecolor{darkblue}{rgb}{0,0,0.5}

% Set hyperlink settings
\hypersetup{ 
	colorlinks = true, 
	linkcolor=darkblue, 
	filecolor=darkgreen, 
	urlcolor=darkred , 
	citecolor=darkblue 
}



% \bibpunct{[}{]}{;}{a}{,}{,}

\title{\thesisTitle}
\author{\thesisAuthor}
\date{\today}

\begin{document}
\begin{titlepage}
\noindent {\large \textbf{\thesisAuthor}}
\vspace{2cm}

\noindent {\Huge \thesisTitle}
\vspace{2cm}

\noindent \thesisType, \thesisDate \footnote{http://www.idi.ntnu.no/emner/tdt4501}
\vspace{2cm}

\noindent Department of Computer and Information Science\\ Faculty of Information Technology, Mathematics and Electrical Engineering\\

\vfill
\begin{center}
\includegraphics[width=3cm]{figs/NTNUlogo.pdf}
\end{center}
\end{titlepage}

\thispagestyle{empty}

\cleardoublepage

\frontmatter

\section*{Abstract}

Sentiment Analysis (or opinion mining) on Twitter has become very popular in later years. Sentiment and opinion data has a wide range of applications of in fields like marketing research, end-user product research, reviews and more. This has made it a very appealing topic for research.

In the report a systematic literature review is conducted to define the current state-of-the-art for sentiment analysis systems on Twitter data, and an architecture is proposed for building such a system. The proposed system works as a layer on top of the Twitter API, and as such is thoroughly documented, and a lot of developers are already familiar with it. The API layer and the sentiment classification system works in a heavily asynchronous manner which allows for parallel operations and a horizontally scalable system. 

A basic sentiment analysis system is implemented, to work as a proof of concept for the proposed architecture. The classification system is implemented with the use of the AFINN lexicon, specifically designed for sentiment analysis on Twitter.

\clearpage

\section*{Preface}

This specialization project is conducted as a part of a Master's thesis at the Norwegian University of Technology (NTNU) in Trondheim. We have used this project as a mean of gaining sufficient knowledge about and defining the state-of-the-art for sentiment analysis systems. This state-of-the-art survey is to be used as a basis for our Master's thesis in the last semester of our studies at NTNU. 


We used a systematic literature review to gain knowledge and finding research papers. This was a new task for us, and we found it useful and educational, but a very demanding process. 


We would like to thank our supervisor Björn Gambäck, and assisting supervisors Amitava Das and Lars Bungum for interesting feedback and for always being engaged in our project. Björn Gambäck always had educational papers on hand and helped us grasp the topic at hand.


\vfill

\hfill \thesisAuthor

\hfill Trondheim, \today

\clearpage

\tableofcontents

\listoffigures

\listoftables

\mainmatter
\chapter{Introduction}
\section{Task Description}
\label{sec:task}
The task was given by Björn Gambäck and Lars Bungum at IDI, NTNU:
\begin{center} \Large Sentiment Analysis using the Twitter Corpus \end{center}
\begin{quotation}
In recent years, micro-blogging has become prevalent, and the Twitter API allows users to collect a corpus from their micro-blogosphere. The posts, named tweets are limited to 140 characters, and are often used to express positive or negative emotions to a person or product.

In this project, the goal is to use the Twitter corpus to do sentiment analysis. Pak and Paroubek (2010) have shown how to do this using frameworks like Support Vector Machines (SVMs) and Conditional Random Fields (CRFs), benchmarked with a Naive Bayes Classifier baseline. They were unable to beat the baseline, and the goal of this project will be to experiment with these and other machine learning frameworks as Maximum Entropy learners to try to beat the baseline.
\end{quotation}

\section{Motivation}
The growth in Twitter users and status updates (tweets) over the last years has made Twitter an attractive platform for companies, marketeers, politicians and others who are looking for feedback. Manually collecting information like this is a tedious if not impossible task.

The informal texts on social media represent challenges for traditional natural language processing systems. These texts are short, and  often contain misspellings, slang and abbreviations. The challenge of handling such a vocabulary has only been researched over the last few years.

Another interesting feature is that Twitter messages offer a lot of meta data and information about their origin, such as location, language, and more. These data could for example be used to filter out and classify tweets from a certain event, like a festival or a conference.

\section{Project goals}
In this section the main goals for this project are described. As the assignment~\autoref{sec:task} was intended for a Master's thesis, this specialization project is used as a pre-study.

	\subsection{Research on state-of-the-art sentiment analysis}
	A lot of work has already been done in the field of sentiment analysis, also when using the Twitter corpus. To be able to make a contribution to this work, we have to do research to gather knowledge about existing solutions and their performance.
	
	\subsection{Establish future work}
	As this is a pre-study project for a Master's thesis, one of the goals is to establish tasks for the thesis. Future work, as defined in this report, may be used as a baseline for the project goals in the Master's thesis.
	
	\subsection{Server for sentiment classification}
	Design and implement a highly modular python server with a basic form of classification. This system will work as a foundation for implementing the complete classification system in the up-coming master thesis. For that reason it should be as modular as possible, to make it easy to replace different parts of the system when necessary.

	\subsection{API layer with Twitter API integration}
	Twitter offers a well documented Representational State Transfer (REST) Application Programming Interface (API) to obtain data from their corpus. To make our system easy to use for developers already using the Twitter platform, an API that is compatible with the Twitter API should be implemented. As an extension to the output of the already existing API, a \emph{sentiment} attribute will be added to the returned tweet. This attribute should hold the result of the sentiment classification.
	

\section{Twitter}

Twitter has become a popular social media service often referred to as a micro-blogging site. On Twitter users can post messages of maximum 140 characters, called tweets, on their own \emph{timeline}. A timeline is a collection of all user submitted tweets and all tweets from the other users that a user is connected to (following). Tweets can be categorized by using hashtags. A hashtag can, for instance, look like \emph{\#happy} or \emph{\#obama2012}. By annotating the tweets with this tag, users can find similar tweets across Twitter.

Twitter has grown very rapidly and the usage statistics is ever changing. In June 2012, there were posted over 400 million tweets every day~\citep{site:twitterusage}. With over 500 million users, where about 170 million of these are active ones~\citep{site:users}, it is safe to say that Twitter offers a lot of data.


The informal nature of Twitter leads to a lot of sentiments being posted and this has led Twitter to being a gold mine for SA. Many systems have used Twitter as corpus for sentiment analysis. The first one to really use Twitter as a corpus was Sentiment140 (previously known as TwitterSentiment) by a group of Stanford students~\citep{article:go}. After this paper, and as Twitter grew in popularity, many other systems have been developed. Some of the later ones are TwiSent and C-Feel-IT~\citep{mukherjee2012twisent}, Tweenator~\citep{saif2012semantic} and MSAS~\citep{chamlertwat2012discovering}.

\chapter{Background}
\textbf{SOME INTRODUCTORY DESCRIPTION HERE!}

\section{Twitter API}
Twitter allows developers and others to access their data by the means of an application program interface (API). The API is implemented by using Representational State Transfer (REST). REST is a style of software architecture for distributing data across the World Wide Web (WWW). By using the protocol HTTP's vocabulary of methods, developers can get, insert, delete and update data on Twitter, given the proper access. Some of the important methods are GET for retrieving data, POST for inserting, updating and sending data, DELETE for removing data.~\cite{article:rest}

Twitter uses OAuth for authentication. OAuth provides a way for clients to access resources on behalf of end users or other clients. OAuth is also used to provide third party applications access to a users data on a service, without sharing the users user name or password. \cite{site:oauth}

After migrating to version 1.1 of their API, Twitter now has a limit on the number of requests an end user can do. As end users authenticate using OAuth, Twitter can identify them and limit their access if overused. The limitations are divided into 15 minute intervals. Some services on the API are limited to 15 requests per time window, other services are limited by 180 requests. E.g. the search service is limited by 180 requests, but the service for retrieving tweets from a Twitter list is limited by 15.~\cite{site:twitterlimit}

Almost any aspect of Twitter is covered by the REST API. The most interesting parts are the ones to retrieve tweets in various ways. There are methods for retrieving entire user time-lines, mentions of a user, favourite tweets for a user, tweets based on geographical annotated locations, and so on.~\cite{site:twitterapi}

There are also real time streaming services in the API. Streams are not based on the Twitter REST API. For streaming a persistent HTTP connection is opened to the API. This way a client would not have to continuously poll the REST API to register changes or new data. A client would be provided by real time data from Twitter, but only have one single connection opened.~\cite{site:twitterstream}

The Twitter API uses JavaScript Object Notation (JSON) as their format for response. JSON is a lightweight data format often used as an alternative to XML. JSON was created as a subset of the JavaScript Programming Language but is language-independent. JSON has a simple structure and aims to be minimal, portable and textual and has support for four primitive types (strings, numbers, booleans, and null) and two structure types (array and objects).~\cite{site:json}



\section{Node.js}
Node.js is a platform built on Google's V8 JavaScript Engine. In later years Node.js as grown rapidly in popularity, this much due to it's scalability and event driven nature. As Node.js uses a non-blocking I/O model, it is perfect for data-intensive real-time applications~\cite{site:nodejs}. Node.js is often refereed to as JavaScript on the server-side.	

\section{Machine Learning}
	\subsection{Naive Bayes classifier}
	\subsubsection{Naive Bayes Classifier}
		The naive Bayes classifier (NB) is a practical Bayesian learning model that are easy to understand and implement. For some classification tasks it has proven to be equally performing to more complex machine learning algorithms like artificial neural networks (ANN) and decision trees(DT)[!ref]. NB is used for learning cases where an instance $x$ consists of a number of attribute-value pairs, and the target function $f(x)$ consists of a finite number of values from a set $V$.

The NB classifier is based on an assumption that all the attribute values are conditionally independent given the target value of the instance.
\begin{equation}
\label{equation:nbc}
v_{NB} = P(v_j) \amalg p(a_i|v_j)
\end{equation}

To classify an instance, the classifier use the Maximum Likelihood Estimation (MLE) method to find the ratio of an attribute value and a given target value in the same instance in the training corpus. This means that it has to calculate the probability estimate $P$ for each attribute $a_i$, given the target value. It then assigns the target value as the one that gives the highest product from multiplying all the probabilities $P$ from the training data.
	
	\subsection{Maximum Entropy}
	\input{maxent}

\section{Structured Literature Review}
This section will describe what a systematic literature review (SLR) is and how it is conducted. 

A SLR is a methodical and formal way of retrieving information and doing literature search regarding a topic. SLR has been used to a great extent in fields like medicine, but not as much in computer science. A SLR uses several steps to accomplish the literature search and uses a structured literature review protocol to document these steps. This way the literature search can be reproducible and is documented. \citep{paper:slrdesc}

\subsection{Performing a SLR}

A SLR consists of three phases with the total of 13 steps: planning, conducting and reporting, as defined by \cite{paper:slrdesc} and \cite{master:slr}:

\subsubsection{Planning}

\begin{description}

	\item[1. Identification of the need for a review] \hfill \\
		Identify whether or not the researchers need to review literature using SLR.

	\item[2. Commissioning a review] \hfill \\
		Organizations need to have the resources to do a SLR and commission the review.

	\item[3. Specifying the research question(s)] \hfill \\
		Define research questions (RQ) to specify what the researchers hope to find information about. The RQs is used as basis for, among others, the search, inclusion, and data collection in the conduction phase. 

	\item[4. Developing a review protocol] \hfill \\
		Develop a protocol providing documentation for all steps for conducting a SLR.
	

	\item[5. Evaluating the review protocol] \hfill \\
		Have a independent party or person evaluate the protocol.

\end{description}

% This probably shouldn't be here but in the SLR chapter..
% In this report both step 1 and 2 are assumed completed.

\subsubsection{Conducting}

\begin{description}

	\item[1. Identification of research] \hfill \\
		Locate as many papers as possible relevant to the previously defined and documented RQs. The search strategy needs to be documented with clear definitions for search domains and search string.

	\item[2. Selection of primary studies] \hfill \\
		Use a set of inclusion criteria (IC) to select what papers to use for the review. This can be done in two steps, by filtering on only title and abstract or based on the full text paper. 

	\item[3. Study quality assessment] \hfill \\
		Use a set of quality criteria (QC) to further filter down the paper collection. Use the QCs to set a score of each paper. Set a lower score limit to filter irrelevant or papers of poor quality.

	\item[4. Data extraction and monitoring] \hfill \\
		Extract data from All the papers selected and included. This data collection should be documented by the protocol, defining what features and information to extract.
	

	\item[5. Data synthesis] \hfill \\
		Synthesise the data to be able to answer the RQs. 
\end{description}

% This probably shouldn't be here but in the SLR chapter..
% For this report, the data synthesis is included as a part of the data extraction. 

\subsubsection{Reporting}


\begin{description}

	\item[1. Specifying dissemination strategy] \hfill \\
		Specify how the review result shall be presented. 

	\item[2. Formatting the main report] \hfill \\
		Report the literature review in it's own report or as a part of a report.

	\item[3. Evaluating the report] \hfill \\
		Have an independent expert in the field evaluate the SLR. 

\end{description}



\chapter{Related Work}
In this section it is described how the structured literature review was conducted and the result of the review. The first section, \autoref{sec:slrintro}, covers the introduction and defines our research questions. The review method is described in \autoref{sec:slrmethod} and the result in \autoref{sec:slrresults}.


\section{Introduction}
\label{sec:slrintro}

SLRs are still new in the field of computer science. There are few examples of a SLR in practise. The method we used is heavily inspired from the documentation paper by~\cite{paper:slrdesc} and the Master's thesis by~\cite{master:slr}. \\

\noindent We defined our research questions as the following:

\begin{description}

\item[RQ1] What are some of the existing solutions for SA (sentiment analysis) in the Twitter Corpus.
\item[RQ2] How does the different solutions found by addressing RQ1 compare to each other with respect to micro-blogs like Twitter.
\item[RQ3] What is the strength of the evidence in support of the different solutions?
\item[RQ4] What implications will these findings have when creating the application/system?

\end{description}

\section{Review method}
\label{sec:slrmethod}

In this section we will describe our process step by step according to the structured literature review. 

\subsection{Planning}

\begin{description}

	\item[1. Identification of the need for a review] \hfill \\
		Assumed that a review is needed for this project. 
		
	\item[2. Commissioning a review] \hfill \\
		Assumed to be commissioned for this project, so no commission report was produced. 

	\item[3. Specifying the research question(s)] \hfill \\
		We defined the RQs based on what we felt needed to be researched when finding the state-of-the-art for sentiment analysis systems on Twitter. The RQs can be seen in \autoref{sec:slrintro}. 

	\item[4. Developing a review protocol] \hfill \\
		The review protocol was developed in the early stages of the project. After first revision it was evaluated by the project supervisor. The protocol was revised several times during the execution of the review.
	

	\item[5. Evaluating the review protocol] \hfill \\
		The protocol was evaluated by the project supervisor. 

\end{description}


\subsection{Conducting}

\begin{description}

	\item[1. Identification of research] \hfill \\
		We defined a series of keywords and synonyms to construct a search string to use. The search string was defined to find papers relevant to our research questions. Development of this search string is documented by~\autoref{apx:slrp}. The search string we used was:
		
		\begin{verbatim}
		("Sentiment Analysis" OR "Sentiment Classification" 
		OR "Opinion Mining") AND (Twitter OR Microblog)
		\end{verbatim}
		
		For the search domain, we used Google Scholar. Google Scholar accumulates results from several different sources and gave many results for our search. Many of the results given corresponded with the studies handed out by our supervisor. 
		
		We limited the search to only give papers released after 2008. This is due to Twitter being as new as it is.
		
		The search resulted in 1060 papers, but after the first 8 pages of results (with 10 papers per page), we found that the papers got mostly irrelevant and we had limited resources and time for handling all 1060 papers. This resulted in a set of 80 papers, ready to go through the selection process.
		
		

	\item[2. Selection of primary studies] \hfill \\
		Firstly we defined a set of primary inclusion criteria. These criteria were applied to the title and abstract of the study. If a paper did not pass the criteria, it was dropped from the set. Secondly we defined secondary inclusion criteria. These criteria were checked up against the full text of the study. These inclusion criteria was documented in the protocol in~\autoref{apx:slrp}. 
		
		After both selection processes, we had a set of 23 papers. 

	\item[3. Study quality assessment] \hfill \\
		With the help of \cite{paper:slrdesc} we defined a set of 10 quality criteria. All of the criteria was documented in the protocol in~\autoref{apx:slrp}. Each of the papers in our set was assessed with all of these criteria. If the paper met the criteria, it would get 1 point, if met partly it would get 0,5 points, and if it did not meet the criteria it would get 0 points. 
		
		The papers with the lowest score did not get taken as much into consideration when defining the state-of-the-art. 
		
	\item[4. Data extraction and monitoring] \hfill \\
		We defined a set of fields and information categories we thought were important in order to answer our research questions. These data features were used to generate a table of information. The information was retrieved by reading the papers and manually extracting the data we needed.
	

	\item[5. Data synthesis] \hfill \\
		This step was included in the data extraction step.
\end{description}

% This probably shouldn't be here but in the SLR chapter..
% For this report, the data synthesis is included as a part of the data extraction. 

\subsection{Reporting}


\begin{description}

	\item[1. Specifying dissemination strategy] \hfill \\
		As this is for a specialisation project, the result of the SLR is presented in this report.  

	\item[2. Formatting the main report] \hfill \\
		The SLR was written as a section in this specialisation project report. 

	\item[3. Evaluating the report] \hfill \\
		It is mandatory for an expert to review this report as well as a project supervisor, as it is a report for the specialization project.

\end{description}

\section{Results}
\label{sec:slrresults}

In this section the result of the systematic literature review is presented. In \autoref{sec:selected} all of the extracted data from the selected studies are presented. The assessed quality is also presented. 

In \autoref{sec:stateofart} the result to our research questions are defined. 

\subsection{Selected studies}
\label{sec:selected}

In this section all of the selected studies are presented in table form as a part of the data extraction step in the SLR. The results are split into four different tables. In addition to the data features defined in the review protocol, the total quality is added to be a part of the extraction table.


\newcolumntype{P}[1]{>{\raggedright\arraybackslash}p{#1}}

\begin{landscape}
    \centering
    \tablefirsthead{%
    	\hline
		\textbf{ID} & \textbf{Authors} & \textbf{Title} & \textbf{Pub. Year} & \textbf{System name} & \textbf{ML Algorithm} & \textbf{Dataset} & \textbf{Findings} \& \textbf{Conclusions} & \textbf{QA}\\
		\hline
	}
	
	\tablehead{%
		\hline
		\multicolumn{9}{|c|}{Continued from previous page}\\
		\hline
		\textbf{ID} & \textbf{Authors} & \textbf{Title} & \textbf{Pub. Year} & \textbf{System name} & \textbf{ML Algorithm} & \textbf{Dataset} & \textbf{Findings} \& \textbf{Conclusions} & \textbf{QA}\\
		\hline
	}
	\tabletail{%
		\hline
		\multicolumn{9}{|c|}{Continuing next page}\\
		\hline
	}
	\tablelasttail{\hline}
	
	\tablecaption{Data extraction step. Showing data as per defined in the SLRP in~\autoref{apx:slrp}}
	\label{tab:extraction}
	
	\scriptsize
	\begin{supertabular}{|P{0.5cm}|P{2.8cm}|P{3.5cm}|P{0.6cm}|P{1cm}|P{1cm}|P{3.3cm}|P{3cm}|P{0.5cm}|}
	
	  
		S1 & Hassan Saif, Yulan He \& Harith Alani & Semantic Smoothing for Twitter Sentiment Analysis & \citeyear{saif2011semantic} & - & NB & http://twittersentiment. appspot.com/ & Slight improvement by .3\% & 7.5 \\ \hline  
		
		S2 & Subhabrata Mukherjee, Akshat Malu, A.R. Balamurali, Pushpak Bhattacharyya & TwiSent: A Multistage System for Analyzing Sentiment in Twitter & \citeyear{mukherjee2012twisent} & TwiSent & NB & C-Feel-IT dataset.  & Improvements with filtering/ normalization. Best: 66,69\% with manually annotated dataset  & 8.5 \\ \hline  
		
		S3 & Adam Bermingham \& Alan Smeaton & Classifying Sentiment in Microblogs:Is Brevity an Advantage? & \citeyear{article:bermingham} & - & NB, SVM & CLARITY dataset (removed due to Twitter terms), Pang \& Lee's movie corpus,TREC Blogs06 corpus and a collection of microreviews from Blippr & Accuracy of 74.85\% using NBC and unigrams. Easier to classify microblogs than long texts  & 9.0 \\ \hline  
		
		S4 & Wilas Chamlertwat, Pattarasinee Bhattarakosol, Tippakorn Rungkasiri, Choochart Haruechaiyasak & Discovering Consumer Insight from Twitter via Sentiment Analysis & \citeyear{chamlertwat2012discovering} & MSAS & SVM, Lexical (non ML) & Self compiled, manually annotated 600 tweets & Sentiment Analysis can be useful for consumer research. No accuracy for classification.  & 8.0 \\ \hline  
		
		S5 & Dmitry Davidov, Oren Tsur \& Ari Rappoport & Enhanced Sentiment Learning Using Twitter Hashtags and Smileys & \citeyear{davidov2010enhanced} & - & HFW/ CW & Dataset by Brendan O?Connor. Hashtags and smileys as training labels. & Hashtags and smilies works good for SA  & 6.5 \\ \hline  
		
		S6 & Murphy Choy, Michelle L.F. Cheong, Ma Nang Laik \& Koo Ping Shung & A sentiment analysis of Singapore Presidential Election 2011 using Twitter data with census correction & \citeyear{choy2011sentiment} &  &  &  & Given proper recalibration using census information, the twitter information can translate into pretty accurate information about the political landscape.  & 5.5 \\ \hline  
		
		S7 & Hassan Saif, Yulan He \& Harith Alani & Semantic Sentiment Analysis of Twitter & \citeyear{saif2012semantic} & Tweentor & NB & Stanford Twitter Sentiment Corpus, Health Care Reform, Obama-McCain Debate & Improvements by using semantic features on wide range topics. Acc: 83.9\%  & 9.0 \\ \hline  
		
		
		S8 & Lei Zhang, Riddhiman Ghosh, Mohamed Dekhil, Meichun Hsu \& Bing Liu & Combining Lexicon-based and Learning-based Methods for Twitter Sentiment Analysis & \citeyear{zhang2011combining} &  & SVM, Lexical &  & Outperforms approaches that use lexical or supervised methods alone  & 8.5 \\ \hline  
		
		S9 & Alec Go, Lei Huang, Richa Bhayani & Twitter Sentiment Analysis & \citeyear{article:go} & Senti-ment140 & NB, SVM & http://www.stanford.edu/ ~alecmgo/cs224n/ twitterdata.2009.05.25.c.zip & ~85\% - bias accuracy.  & 9.0  \\ \hline  
		
		
		S10 & James Spencer \& Gulden Uchyigit & Sentimentor: Sentiment Analysis of Twitter Data & \citeyear{spencer2012sentimentor} & Senti-mentor & NB &  & 52\% for three classes(pos,neg and objective) using bigrams without POS tagging  & 7.5 \\ \hline  
		
		
		S11 & Amir Asiaee T, Arindam Banerjee, Mariano Tepper \& Guillermo Sapiro  & If You are Happy and You Know It... Tweet & \citeyear{asiaee2012if} &  & NB, SVM, k-NN, DL &  & 82.95\% accuracy with NB on tweets about the weather  & 8.5 \\
		
		S12 & Finn Arup Nielsen & A new ANEW: Evaluation of a word list for sentiment analysis in microblogs & \citeyear{article:afinn} &  & Lexical & Labeled language data created with Amazon Mechanical Turk(AMT) & The AFINN word list performs slightly better than ANEW in Twitter sentiment analysis  & 7.5 \\ \hline  
		
		S13 & Luciano Barbosa \& Junlan Feng & Robust sentiment detection on Twitter from biased and noisy data & \citeyear{barbosa2010robust} & TwitterSA & SVM & Used Twendz, Twitter Sentiment and TweetFeel to collect data & By using data with noisy labels as input, they achieved a more abstract representation of Twitter messages than raw words.  & 8.5 \\ \hline
		
		S14 & Younggue Bae \& Hongchul Lee & A Sentiment Analysis of Audiences on Twitter: Who Is the Positive or Negative Audience of Popular Twitterers? & \citeyear{bae2011sentiment} &  & Used LIWC-2007 dictionary to extract sentiment & Collected tweets from celebrities and their mentions & & 5.5  \\ \hline  
		    
		S15 & Mauro Cohen, Pablo Damiani, Sebastian Durandeu, Renzo Navas, Hern\'{a}n Merlino, Enrique Fern\'{a}ndez & Sentiment Analysis in Microblogging: A Practical Implementation & \citeyear{cohen2011sentiment} & - & NB & Manually gathered and annotated. 1500 tweets & Not as good accuracy as previous systems & 4.0 \\ \hline  
		
		S16 & Roberto Gonz\'{a}lez-Ib\'{a}\~{n}ez, Smaranda Muresan, Nina Wacholder & Identifying Sarcasm in Twitter: A Closer Look & \citeyear{gonzalez2011identifying} & - & SMO, LogR & Data collected by using hashtag search. 900 tweets & Best result (75.89\%) was achieved in the polarity- based classification P-N. Automatic classification can be as good as human classification; however, the accuracy is still low & 9.5 \\ \hline  
		
		S17 & Akshi Kumar, Teeja Mary Sebastian & Sentiment Analysis on Twitter & \citeyear{kumar2012sentiment} & - & None. POS and own alg. &  & The initial results show that it is a motivating technique. No stated accuracy & 7.5 \\ \hline  
		
		S18 & Alexander Pak, Patrick Paroubek & Twitter as a Corpus for Sentiment Analysis and Opinion Mining & \citeyear{article:pak} & - & NB & http://www.stanford.edu/  ~alecmgo/cs224n/ twitterdata.2009.05.25.c.zip & ~63\% accuracy using bigrams and POS tagging & 10 \\ \hline 
		
		S19 & Long Jiang, Mo Yu, Ming Zhou, Xiaohua Liu, Tiejun Zhao & Target-dependent Twitter Sentiment Classification & \citeyear{article:jiang} & - & SVM & Subjectivity: Manually annotated 727 tweets for each class. Polarity: Manually annotated 268 tweets for each (pos, neg) & Subjectivity: 85.6\%. Polarity: 68.2\% & 8.5 \\ \hline  
		    
		S20 & Kun-Lin Liu, Wu-Jun Li \& Minyi Guo & Emoticon Smoothed Language Models for Twitter Sentiment Analysis & \citeyear{liu2012emoticon} & ESLAM & Unigram, MLE & Sanders Corpus (5513 manually labeled tweets with one of the four different topics: Apple, Google, Microsoft, and Twitter) & ESLAM performs better than both emoteicons(distant supervised) and manually annotated tweets(fully supervised) alone & 10 \\
		
		S21 & Efthymios Kouloumpis, Theresa Wilson, Johanna Moore & Twitter Sentiment Analysis: The Good the Bad and the OMG! & \citeyear{article:omg} & - & AdaBoo-st.MH & Gathered by hash, http://twittersentiment. blogspot.com and iSieve Corporation for evaluating data & Hash + Emoticons result in 75\% accuracy & 8.0 \\ \hline  
		
		S22 & Apoorv Agarwal, Boyi Xie, Ilia Vovsha, Owen Rambow, Rebecca Passonneau & Sentiment Analysis of Twitter Data & \citeyear{vovsha2011sentiment} & - & SVM & Data by NextGen Invent (NGI). Manually annotated 11,875 tweets, non-bias tweets & A gain of 4\% on unigram 3-way classification. Acc: 60.83\% & 10 \\ \hline  
		
		S23 & Asli Celikyilmaz, Dilek Hakkani-T\"{u}r \& Junlan Feng  & Probabilistic Model-based Sentiment Analysis Of Twitter Messages & \citeyear{celikyilmaz2010probabilistic} &  & Used LDA to extract a polarity lexicon & Collected 2 million tweets using Twitter search API from September 2009 to June 2010. Keywords related to mobile operation. Made two manually annotated subsets from this collection. & Relatively 10\% better F-measure with unigrams than baseline for classification with text normalization and all word unigram, bigram and trigrams as features. & 7.5 \\
		
	\end{supertabular}
\end{landscape}

In \autoref{tab:quality} all the individual points for the quality criteria is presented. The criteria is defined as a part of the review protocol.

\begin{table}
    \centering
    \scriptsize
    
    \setlength\tabcolsep{2pt}
    \def\arraystretch{1.4}%  1 is the default, change whatever you need
    \begin{tabular}{|c|c|c|c|c|c|c|c|c|c|c||c|} 
    
    \hline
    \textbf{\#ID} & \textbf{QC1} & \textbf{QC2} & \textbf{QC3} & \textbf{QC4} & \textbf{QC5} & \textbf{QC6} & \textbf{QC7} & \textbf{QC8} & \textbf{QC9} & \textbf{QC10} & \textbf{Total} \\
    \hline
    
    \textbf{S1} & 1,0 & 0,5 & 1,0 & 0,5 & 1,0 & 1,0 & 0,5 & 0,5 & 0,5 & 1,0 & 7,5 \\ \hline
    \textbf{S2} & 1,0 & 1,0 & 1,0 & 0,5 & 1,0 & 1,0 & 1,0 & 0,5 & 0,5 & 1,0 & 8,5 \\ \hline
    \textbf{S3} & 1,0 & 1,0 & 0,5 & 0,5 & 1,0 & 1,0 & 1,0 & 1,0 & 1,0 & 1,0 & 9,0 \\ \hline
    \textbf{S4} & 1,0 & 1,0 & 0,5 & 0,5 & 1,0 & 1,0 & 1,0 & 0,5 & 0,5 & 1,0 & 8,0 \\ \hline
    \textbf{S5} & 1,0 & 0,5 & 0,5 & 0,5 & 1,0 & 1,0 & 0,0 & 0,5 & 1,0 & 0,5 & 6,5 \\ \hline
    \textbf{S6} & 1,0 & 0,5 & 0,5 & 0,5 & 0,0 & 0,5 & 0,0 & 0,5 & 1,0 & 1,0 & 5,5 \\ \hline
    \textbf{S7} & 1,0 & 1,0 & 1,0 & 0,5 & 0,5 & 1,0 & 1,0 & 1,0 & 1,0 & 1,0 & 9,0 \\ \hline
    \textbf{S8} & 1,0 & 1,0 & 0,5 & 1,0 & 1,0 & 1,0 & 1,0 & 0,5 & 0,5 & 1,0 & 8,5 \\ \hline
    \textbf{S9} & 1,0 & 1,0 & 1,0 & 0,5 & 1,0 & 1,0 & 1,0 & 1,0 & 0,5 & 1,0 & 9,0 \\ \hline
    \textbf{S10} & 1,0 & 1,0 & 0,5 & 1,0 & 1,0 & 1,0 & 0,5 & 0,5 & 0,5 & 0,5 & 7,5 \\ \hline
    \textbf{S11} & 1,0 & 0,5 & 1,0 & 0,0 & 1,0 & 1,0 & 1,0 & 1,0 & 1,0 & 1,0 & 8,5 \\ \hline
    \textbf{S12} & 1,0 & 1,0 & 0,0 & 1,0 & 0,5 & 1,0 & 0,0 & 1,0 & 1,0 & 1,0 & 7,5 \\ \hline
    \textbf{S13} & 1,0 & 1,0 & 1,0 & 1,0 & 0,5 & 0,5 & 0,5 & 1,0 & 1,0 & 1,0 & 8,5 \\ \hline
    \textbf{S14} & 0,5 & 0,5 & 0,5 & 1,0 & 0,0 & 0,5 & 0,5 & 0,5 & 0,5 & 1,0 & 5,5 \\ \hline
    \textbf{S15} & 1,0 & 0,0 & 0,0 & 0,5 & 1,0 & 1,0 & 0,0 & 0,0 & 0,0 & 0,5 & 4,0 \\ \hline
    \textbf{S16} & 1,0 & 1,0 & 0,5 & 1,0 & 1,0 & 1,0 & 1,0 & 1,0 & 1,0 & 1,0 & 9,5 \\ \hline
    \textbf{S17} & 1,0 & 1,0 & 1,0 & 1,0 & 1,0 & 1,0 & 0,5 & 0,5 & 0,0 & 0,5 & 7,5 \\ \hline
    \textbf{S18} & 1,0 & 1,0 & 1,0 & 1,0 & 1,0 & 1,0 & 1,0 & 1,0 & 1,0 & 1,0 & 10,0 \\ \hline
    \textbf{S19} & 1,0 & 1,0 & 0,5 & 0,5 & 1,0 & 1,0 & 0,5 & 1,0 & 1,0 & 1,0 & 8,5 \\ \hline
    \textbf{S20} & 1,0 & 1,0 & 1,0 & 1,0 & 1,0 & 1,0 & 1,0 & 1,0 & 1,0 & 1,0 & 10,0 \\ \hline
    \textbf{S21} & 1,0 & 1,0 & 1,0 & 0,5 & 1,0 & 1,0 & 0,5 & 0,5 & 0,5 & 1,0 & 8,0 \\ \hline
    \textbf{S22} & 1,0 & 1,0 & 1,0 & 1,0 & 1,0 & 1,0 & 1,0 & 1,0 & 1,0 & 1,0 & 10,0 \\ \hline
    \textbf{S23} & 0,5 & 1,0 & 1,0 & 0,5 & 0,5 & 0,5 & 0,5 & 1,0 & 1,0 & 1,0 & 7,5 \\ \hline
     &  &  &  &  &  &  &  &  &  & \textbf{Avg} & \textbf{8,00} \\ \hline
    
    \end{tabular}
    
    \caption{Quality Assessment for the studies. Each QC can give 0, 0.5 or 1 point. The average score is 8.}
    \label{tab:quality}
\end{table}

\subsection{State-of-the-art}
\label{sec:stateofart}



\chapter{Architecture}
In this section we will describe the overall architecture and how the system works. First the general system will be described, and then the API Layer and Classification Server in turn.  

To make the system as modularized and responsive as possible, the API layer was written in Node.js while the sentiment classification in the Python programming language. Both systems are continuously running servers. This allows us to have multiple services running, both for the API layer and the classifier, if needed for horizontal scalability. 

\insertfig[0.8]{NetworkDiagram.pdf}{Architectural overview of the system. Client retrieves data from the Twitter API and uses the classification server to classify for sentiments.}

A client makes a request to the API Layer, with the same interface as the Twitter API service. From there the API Layer will retrieve information from the Twitter API with HTTP requests, and iterate over all tweets received and send them in parallel to the classification server. When the classification server is done processing and classifying the tweet, it is sent back to the API Layer. When the API layer has received all the tweets, it responds the client with the same JSON structure as the Twitter API sends out, only with an additional attribute noting the tweets sentiment. This architecture and application flow can be seen in~\autoref{fig:NetworkDiagram.pdf}. 

\section{API Layer Extension}


\insertfig[1]{APILayerArcitechture.pdf}{Architectural overview of the API Layer. A request is handled by the server, sending it to routing where it is processed and sent to service look-up. If a service is found a request is sent to the Twitter API and the received data is extended to contain a sentiment by the Twitter Data Handler module. When all of the Twitter data is extended, the data is given as a response to the requesting client.}


To be able to have a scalable and responsive solution, the API Layer was written using the Node.js platform. Since Node.js uses JavaScript as programming language, the JSON data retrieved from the Twitter REST and Streaming API is easily manipulated and passed around. 

The API layer works a thin layer extending the Twitter API. This means that the interface used by Twitter, with all defined options and appropriate methods, is reflected the same through the API Layer. This way all documentation for the Twitter API also documents our API Layer. 



When a request from a client is made, the request gets processed by the server and the routing module determines what the client is looking for. When the proper service is found the client specified parameters is sent directly to the Twitter API, using the Twitter Data Handler module (TDH). The TDH module then iterates over all found tweets, and sends them in parallel to the classification server. When a tweet is processed by the classification server the classified sentiment is sent back to the TBH module and the original tweet object is extended to contain a property with the sentiment. When all tweets are classified, the TBH module passes the extended twitter data to the render module. The render module renders the JSON data and sends it to the client with appropriate HTTP headers set. This application flow can be seen in \autoref{fig:APILayerArcitechture.pdf}.

If there is an error during any part process the error is caught by the routing module, and the error is rendered as a JSON object, in the same manner as it would be by the Twitter API. 



When using both the Twitter REST API and Streaming API, there is a high level of asynchronism. Especially when streaming, it is impossible to predict when the next tweet is received. Due to this the system designed needs to be able to handle this dynamic data flow. Node.js is an event-driven platform, and has a natural support for asynchronous data. 

Every internal and external message passing in the API Layer is asynchronous. When requesting Twitter for data, an event is triggered when that data is ready. In this event all tweets are separately sent to the classifier. By sending all tweets separately in parallel, classification of the entire set of tweets does not take much longer than classifying only one tweet. 

When streaming the TBH module opens a connection to the Twitter APi, but never close it. There is a continuously open connection to the Twitter server, which is feeding the TBH module with single tweets as they get stored in the Twitter system. From the first received tweet, a connection to the requesting client is opened by the render response module. This connection is also never closed. This way there is an open connection between the client and the API Layer and between the API Layer and the Twitter API. The API Layer works as a middleman, taking in tweets, classifying them, and streaming them to the client. By having this entire process asynchronous, the system can process data independently of when it is published.


\section{Sentiment Analysis Classifier}




\chapter{Conclusion}

\section{Discussion}

For this project we introduced three main goals:
\begin{itemize}
\item Establish the state-of-the-art for sentiment analysis systems on Twitter
\item Developed a way to distribute sentiment classifications.
\item Develop an architecture for a sentiment analysis system by implementing a basic system.
\end{itemize}

\noindent In the first section of this chapter we will discuss whether we reached our first goal. In the second section we will discuss more about the second and third goals. 

In the last section we will discuss our experiences with conducting a systematic literature review. 


\subsection{Establishing the state-of-the-art}
From the results of our systematic literature review, we have seen that a lot of experiments have been done in the field of Twitter Sentiment Analysis over the last years. We had little trouble finding enough studies to establish the state-of-the-art. To limit the number of papers, the articles from before 2008 were filtered out.

Establishing the performance of state-of-the-art TSA systems was not a simple task. The different experiments have used different data and test sets, and some have used domain specific data while others have trained their learners for cross-domain classification. Another problem that has to be faced is the time consuming task of manually labeling tweets for training and testing. As a solution to this, researchers have experimented with automatic labeling data sets with the usage of noisy data that can lead to a training set that is biased towards these noisy features. This will in turn affect the accuracy of the system, and it will not be a realistic measure. 

\subsection{Distributing sentiments and system architecture}

When developing a way of distributing the sentiment data, we wanted a way that was natural for developers so that they would easily be able to understand our API and start using it straight away. API development is a demanding process, and many teams devote a lot of time to finding the best way of doing it. Instead of trying to define our own interface from scratch, we decided that reflecting the existing Twitter API was the best way to go. By doing this, we also have all of the documentation of the API predefined and would not have to use time and resources of doing this ourselves. 

The only problem we had with extending the Twitter API was the OAuth protocol. This protocol has a lot of documentation and it can be quite advanced. We first tried to make our API Layer just pass on the authentication data (API key and secret) to Twitter API, without us meddling with it. We found out that we would have to implement an OAuth server ourselves and send the key and secret on to Twitter through that server. This was a task we did not have the time to execute. 

The API Layer is designed to work asynchronously and be very scalable. In theory it should also be that. We have not done as much testing as we have liked to do. If we had had more time we would have stress-tested the system with more simultaneously requesting clients. Both for the API Layer and the classification server. 

By using only asynchronously requests and message sending, all communication between the API Layer and the Sentiment classification server happens in parallel. This means that even though there are 20 tweets lined up for classification, the total time will seem be only somewhat higher than the time for classifying one. The CPU will not be able to handle more parallel operations than it has cores. So while it seems like the processing is carried out in parallel, it is not entirely so. Given enough tasks the CPU will queue up the processes. This is still a lot better than doing all classifications sequentially. 

We designed an architecture for doing sentiment classification based on other existing systems. We feel that the architecture is robust and as modular as we want it to be. However, since we have not implemented an advanced sentiment classifier yet, we are unsure of whether or not the architecture will hold if, scaling up the system. There might be, if done further work on the sentiment analysis system, some changes should be made to the overall architecture. 


\subsection{Systematic Literature Review}

By performing an SLR we found that it was a good process for documenting our literature search and had some very good techniques for deciding what papers to include and build our work on. We also found that carrying out a systematic literature review is a very time consuming process. There are few examples of how to do a systematic literature review in the computer science field, which, at first, made it hard to get a working knowledge of how to perform one. In a perfect situation we would re-do some of our processes after gaining more knowledge of how to conduct a review. This is something we did not have the time to do. 

We feel that we have used the SLR process to such a degree that we have gained sufficient knowledge to define the state-of-the-art for a sentiment analysis system on Twitter data. 

\section{Contributions}
For this project the main goal was to uncover the state-of-the-art for Sentiment Analysis systems by doing a systematic literature review, implementing a way to distribute sentiment data for tweets, and define an architecture for doing sentiment analysis. 

We conducted a SLR and defined the state-of-the-art for SA systems. We also implemented a system for extending the existing Twitter API to contain information about the sentiment of a tweet. All communication in the API Layer is asynchronous and works good for both Twitter data streaming and the REST API. 

The architecture for a sentiment analysis system was defined and a basic form for sentiment classification was implemented. The systems runs as a server, waiting for requests by the API Layer. The API Layer sends a request with an attached serialized tweet JSON object, and expects a string with the classification in return. Each request is handled in its own sub process, allowing for parallel calculation and support the API Layers asynchronous operations. 

\section{Future Work}

\subsection{Sentiment Classifier}
For this project a basic form of a sentiment analysis system was implemented. The main focus for this project was the architecture and defining the state-of-the-art of Twitter Sentiment Analysis systems by doing a systematic literature review (SLR). The SA system was implemented to show a proof of concept for the architecture. In future work this sentiment analysis system should be re-implemented with a more sophisticated approach, using more of the techniques discussed in the SLR. 

\subsection{Visualisation and Summary}

At this time no visualisation or use of the sentiment data is implemented. In future work, several applications using this API layer extension and the sentiment data should be implemented. The proposed API Layer as an extension of the Twitter API should make it easy to use and develop applications using this data. There are no platform dependencies for using this API, so both mobile and desktop applications can be developed. 

\subsection{API Layer}

For now the system described in this report uses its own application key for OAuth. In the future this should be reprogrammed to take in values for application credentials, and pass them on to the Twitter OAuth service. This way clients can use their own API key and only be limited by their own requests. By sharing the API key, every client shares the request quota enforced by Twitter.



\addcontentsline{toc}{chapter}{Bibliography}

\bibliographystyle{plainnat}

\bibliography{bibliography}

\begin{appendices}
  
  
\chapter{Systematic Literature Review Protocol}~\label{apx:slrp}

\section{Introduction}

This SLR protocol was developed during the specialization project of the fall semester 2012. This protocol will be used in both the specialization project and the master thesis. For the fall project this protocol and the SLR in general, was be used for the authors to gain sufficient knowledge about sentiment analysis using the Twitter corpus.  

Twitter is a microblogging platform used by millions of people all over the world. In contrast to other social media platforms, the Twitter messages, called tweets, are limited to a maximum length of 140 characters.

The goal of this fall project is to implement a bare-bone, modular and highly customizable application for doing sentiment analysis on tweets. In addition an extension of the existing Twitter API (Application Programming Interface) will be developed. This will be achieved by mimicking the API interface and passing on the query to the Twitter API. By simply extending the API, the sentiment analysis data will be easy to use for existing Twitter developers, and already be heavily document by the Twitter API team. 

The focus for this SLR is to search for papers with existing solutions for sentiment analysis on the Twitter corpus, in order to uncover the different performances and how the problem has been solved by other researchers. This information will be used to implement a basic sentiment analysis application for the specialization project and to implement a more sophisticated application for the master thesis project. 

\section{Research Questions}

\begin{description}

\item[RQ1] What are some of the existing solutions for SA (sentiment analysis) in the Twitter Corpus.
\item[RQ2] How does the different solutions found by addressing RQ1 compare to each other with respect to micro-blogs like Twitter.
\item[RQ3] What is the strength of the evidence in support of the different solutions?
\item[RQ4] What implications will these findings have when creating the application/system?

\end{description}

\section{Search Strategy}

The domain used for the search will be Google Scholar. Google Scholar aggregates results from different domains, and have some built in functions for searching over synonyms and sorting by citations. 

A set of terms is defined closely based on the first research questions (\textbf{RQ1}). The terms are split into groups where each group consists of words that are synonyms or have similar semantic meaning.

All search terms are placed in a table with the groups as columns and the search term as a row. The entire search string will be constructed by using Boolean notation. All terms in a group are concatenated by the keyword $OR$, and the groups themselves are concatenated by the keyword $AND$. This search string is represented by the following formula: 

\begin{verbatim}
([G1, T1] OR ([G1, T2] OR [G1, T3]) AND ([G2, T1] OR [G2, T2]) 
\end{verbatim}

\begin{table}[htdp]
\begin{center}
\begin{tabular}{|l|l|l|}\hline

& Group 1 & Group 2  \\\hline
Term 1 & Sentiment Analysis & Twitter \\\hline
Term 2 & Sentiment Classification & Microblog \\\hline
Term 3 & Opinion Mining &  \\\hline

\end{tabular}
\caption{Search terms and groupings}
\end{center}
\label{tab:searchterms}
\end{table}

All results found by using the search string, will be collected in a document, and reduced by removing duplicated papers, the same studies published from different sources and studies published before the year 2008.


\section{Selection of primary studies}
To reduce the studies even more, they are assessed using three different screenings; primary, secondary and by quality. The primary and secondary inclusion criteria are used to filter out the non-thematically relevant studies. The primary criterion is used on meta data such as title and abstract, while the secondary is used on the full text paper. The quality screening is also used on the full text as the last step of selection.

\subsection{Primary inclusion criteria}

\begin{description}

\item[IC1] The study’s main concern is Sentiment Analysis.
\item[IC2] The study is a primary study presenting empirical results.
\item[IC3] The study focuses on sentiment analysis on the english language.

\end{description}

\subsection{Secondary inclusion criteria}

\begin{description}


\item[IC4] The study focuses on the Twitter corpus.
\item[IC5] The study describes an implementation of an application.

\end{description}

All studies that make it through the primary and secondary selection criteria, will be passed on to the quality assessments. 

\section{Study quality assessment}

To further filter the papers and assess the quality of the different papers, a set 10 of quality criteria is defined. The first two criteria is used in quality screening, to assess whether the papers include a basic research data.

Each study should be classified according to all 10 quality criteria. They can either be classified as "Yes" (1 point), "Partly" (1/2 point) or "No" (0 points).

\begin{description}

\item[QC1] Is there a clear statement of the aim of the research?
\item[QC2] Is the study put into context of other studies and research?
\item[QC3] Are system or algorithmic design decisions justified?
\item[QC4] Is the test data set reproducible?
\item[QC5] Is the study algorithm reproducible?
\item[QC6] Is the experimental procedure thoroughly explained and reproducible?
\item[QC7] Is it clearly stated in the study which other algorithms the study's algorithm(s) have been compared to?
\item[QC8] Are the performance metrics used in the study explained and justified?
\item[QC9] Are the test results thoroughly analysed?
\item[QC10] Does the test evidence support the findings presented?

\end{description}

\section{Data Extraction}

From each paper, the following data will be extracted for the SLR:

\begin{itemize}

\item Study identifier
\item Name of author(s) 
\item Title
\item Year of publication 
\item Name of system
\item Type of machine learning algorithm
\item Data set source
\item Findings and conclusions

\end{itemize}

The data will be presented in table format. Whereas the data type is divided into columns, and each paper is on its own row.
  
\end{appendices}



\end{document}
