
\chapter{Systematic Literature Review Protocol}~\label{apx:slrp}

\section{Introduction}

This SLR protocol was developed during the specialization project of the fall semester 2012. This protocol will be used in both the specialization project and the master thesis. For the fall project this protocol and the SLR in general, was be used for the authors to gain sufficient knowledge about sentiment analysis using the Twitter corpus.  

Twitter is a microblogging platform used by millions of people all over the world. In contrast to other social media platforms, the Twitter messages, called tweets, are limited to a maximum length of 140 characters.

The goal of this fall project is to implement a bare-bone, modular and highly customizable application for doing sentiment analysis on tweets. In addition an extension of the existing Twitter API (Application Programming Interface) will be developed. This will be achieved by mimicking the API interface and passing on the query to the Twitter API. By simply extending the API, the sentiment analysis data will be easy to use for existing Twitter developers, and already be heavily document by the Twitter API team. 

The focus for this SLR is to search for papers with existing solutions for sentiment analysis on the Twitter corpus, in order to uncover the different performances and how the problem has been solved by other researchers. This information will be used to implement a basic sentiment analysis application for the specialization project and to implement a more sophisticated application for the master thesis project. 

\section{Research Questions}

\begin{description}

\item[RQ1] What are some of the existing solutions for SA (sentiment analysis) in the Twitter Corpus.
\item[RQ2] How does the different solutions found by addressing RQ1 compare to each other with respect to micro-blogs like Twitter.
\item[RQ3] What is the strength of the evidence in support of the different solutions?
\item[RQ4] What implications will these findings have when creating the application/system?

\end{description}

\section{Search Strategy}

The domain used for the search will be Google Scholar. Google Scholar aggregates results from different domains, and have some built in functions for searching over synonyms and sorting by citations. 

A set of terms is defined closely based on the first research questions (\textbf{RQ1}). The terms are split into groups where each group consists of words that are synonyms or have similar semantic meaning.

All search terms are placed in a table with the groups as columns and the search term as a row. The entire search string will be constructed by using Boolean notation. All terms in a group are concatenated by the keyword $OR$, and the groups themselves are concatenated by the keyword $AND$. This search string is represented by the following formula: 

\begin{verbatim}
([G1, T1] OR ([G1, T2] OR [G1, T3]) AND ([G2, T1] OR [G2, T2]) 
\end{verbatim}

\begin{table}[htdp]
\begin{center}
\begin{tabular}{|l|l|l|}\hline

& Group 1 & Group 2  \\\hline
Term 1 & Sentiment Analysis & Twitter \\\hline
Term 2 & Sentiment Classification & Microblog \\\hline
Term 3 & Opinion Mining &  \\\hline

\end{tabular}
\caption{Search terms and groupings}
\end{center}
\label{tab:searchterms}
\end{table}

All results found by using the search string, will be collected in a document, and reduced by removing duplicated papers, the same studies published from different sources and studies published before the year 2008.


\section{Selection of primary studies}
To reduce the studies even more, they are assessed using three different screenings; primary, secondary and by quality. The primary and secondary inclusion criteria are used to filter out the non-thematically relevant studies. The primary criterion is used on meta data such as title and abstract, while the secondary is used on the full text paper. The quality screening is also used on the full text as the last step of selection.

\subsection{Primary inclusion criteria}

\begin{description}

\item[IC1] The study’s main concern is Sentiment Analysis.
\item[IC2] The study is a primary study presenting empirical results.
\item[IC3] The study focuses on sentiment analysis on the english language.

\end{description}

\subsection{Secondary inclusion criteria}

\begin{description}


\item[IC4] The study focuses on the Twitter corpus.
\item[IC5] The study describes an implementation of an application.

\end{description}

All studies that make it through the primary and secondary selection criteria, will be passed on to the quality assessments. 

\section{Study quality assessment}

To further filter the papers and assess the quality of the different papers, a set 10 of quality criteria is defined. The first two criteria is used in quality screening, to assess whether the papers include a basic research data.

Each study should be classified according to all 10 quality criteria. They can either be classified as "Yes" (1 point), "Partly" (1/2 point) or "No" (0 points).

\begin{description}

\item[QC1] Is there a clear statement of the aim of the research?
\item[QC2] Is the study put into context of other studies and research?
\item[QC3] Are system or algorithmic design decisions justified?
\item[QC4] Is the test data set reproducible?
\item[QC5] Is the study algorithm reproducible?
\item[QC6] Is the experimental procedure thoroughly explained and reproducible?
\item[QC7] Is it clearly stated in the study which other algorithms the study's algorithm(s) have been compared to?
\item[QC8] Are the performance metrics used in the study explained and justified?
\item[QC9] Are the test results thoroughly analysed?
\item[QC10] Does the test evidence support the findings presented?

\end{description}

\section{Data Extraction}

From each paper, the following data will be extracted for the SLR:

\begin{itemize}

\item Study identifier
\item Name of author(s) 
\item Title
\item Year of publication 
\item Name of system
\item Type of machine learning algorithm
\item Data set source
\item Findings and conclusions

\end{itemize}

The data will be presented in table format. Whereas the data type is divided into columns, and each paper is on its own row.