\section{Background Theory}
	\subsection{Twitter Sentiment analysis}
	This chapter presents some of the state-of-the-art Twitter Sentiment Analysis(TSA) approaches, and the techniques used in the research behind them. The vocabulary used on Twitter makes it hard for traditional natural language processing systems to understand, because they are usually trained on a formal language. This has made researchers exploit some of the special features that the web language, and Twitter gives us, e.g abbreviations and emoticons.
	
		\subsubsection{Data Collection and Preprocessing}
		Most of the data used in TSA research is collected through the Twitter API, either by searching for a certain topic/keyword or by streaming real-time data. Some datasets from related research on the Twitter platform has also been made available for other research projects, as an alternative to collecting a complete data set from scratch. Some approaches specialize on certain domains, while others query for tweets containing emoticons (\emph{':)', ':)'}) to train a cross-domain classifier~\citep{article:go}. The idea behind the emoticon approach is to make sure that the collected tweets contain subjectivity, but these training sets alone are limited to binary classification only (positive/negative classification).

After the data has been collected it should go through a filtering process. First, all non-English tweets are removed, then the Twitter specific symbols and functions described in \autoref{tab:features} would normally be filtered out. As mentioned, a study by \cite{article:go} used ':)' and ':(' as a label for the polarity in their training data, and thus they did not remove these emoticons, but the URLs and usernames were replaced by a nomial ('URL' or 'USERNAME'). They also removed the query term from the text so that it would not affect the classification.

\cite{article:omg} used a hashtagged data set (HASH) in addition to an emoticon data set (EMOT) from http://sentiment140.com. The hashtagged set is a subset of the Edinburgh Twitter corpus which consists of 97 million tweets~\citep{article:edinburgh}.

\begin{table}[]
\centering
\begin{tabular}{|l|l|p{8cm}|}
\hline
RT & Retweet & Reposting another user’s tweet \\ \hline
@ & Mention & Tag used to mention another user \\ \hline
\# & Hashtag & Hashtags are used to tag a tweet to a certain topic. Have become popular recently, and is also used on other platforms \\ \hline
:),:-),$\wedge\wedge$ & Emoticon & Hashtags are used to tag a tweet to a certain topic. Have become popular recently, and is also used on other platforms \\ \hline
URL & URL & Typically a link to an external resource, e.g a new article or a photo \\ \hline
\end{tabular}
\caption{Features that are usually removed from the tweets.}
\label{tab:features}
\end{table}

Some approaches have also experimented with normalizing the tweets, and removing redundant letters, e.g ''loooove'' and ''crazyyy'', that are often used in tweets. Redundant letters like these are sometimes used to express a stronger sentiment, and it has therefore been experimented with trimming down to one additional redundant letter('loooove' = 'loove' instead of love), so that the stronger sentiment can be taken into consideration by a score/weight adjustment for that feature.

\subsubsection*{Part-of-speech tagging}
Part-of-Speech (\nom{POS}{Part-of-Speech}) tagging is a well-known process for marking the words of a sentence. Adjectives, adverbs and personal pronouns have shown good indicators for subjectivity, which has made POS tagging a good technique for filtering out objective tweets before the polarity classification. Early research on TSA showed that the challenging vocabulary made it harder to tag the tweets with a good accuracy; however, in 2010 \cite{article:gimpel} made a POS tagger that aimed at marking tweets. It performed very well in their experiments (almost 90\% accuracy).

		
	
	
		\subsubsection{Naive Bayes Classifier}
		The naive Bayes classifier(NBC) is a practical Bayesian learning model that are easy to understand and implement. For some classification tasks it has proven to be equally performing to more complex classifiers like artificial neural networks(ANN) and decision trees(DT)[ref]. NBC is used for learning cases where an instance $x$ consists of a number of attribute-value pairs, and the target function $f(x)$ consists of a finite number of values from a set $V$.

The NBC is based on an assumption that all the attribute values are conditionally independent given the target value of the instance.
\begin{equation}
\label{equation:nbc}
v_{NB} = P(v_j) \amalg p(a_i|v_j)
\end{equation}

To classify an instance, the classifier's using the Maximum Likelihood Estimation(MLE) method to find the ratio of an attribute value and a given target value in the same instance in the training corpus. This means that it has to calculate the probability estimate $P$ for each attribute $a_i$, given the target value. It then assigns the target value as the one that gives the highest product from multiplying all the probabilities $P$ from the training data.
	\subsection{Twitter API}
	Twitter allows developers and others to access their data by the means of an application program interface (API). The API is implemented by using Representational State Transfer (REST). REST is a style of software architecture for distributing data across the World Wide Web (WWW). By using the protocol HTTP's vocabulary of methods, developers can get, insert, delete and update data on Twitter, given the proper access. Some of the important methods are GET for retrieving data, POST for inserting, updating and sending data, DELETE for removing data.~\citep{article:rest}

Twitter uses OAuth for authentication. OAuth provides a way for clients to access resources on behalf of end users or other clients. OAuth is also used to provide third party applications access to a users data on a service, without sharing the users user name or password.~\citep{site:oauth}

After migrating to version 1.1 of their API, Twitter now has a limit on the number of requests an end user can do. As end users authenticate using OAuth, Twitter can identify them and limit their access if overused. The limitations are divided into 15 minute intervals. Some services on the API are limited to 15 requests per time window, other services are limited by 180 requests. E.g. the search service is limited by 180 requests, but the service for retrieving tweets from a Twitter list is limited by 15.~\citep{site:twitterlimit}

Almost every aspect of Twitter is covered by the REST API. The most interesting parts are the ones to retrieve tweets in various ways. There are methods for retrieving entire user time-lines, mentions of a user, favourite tweets for a user, tweets based on geographical annotated locations, and so on.~\citep{site:twitterapi}

There are also real time streaming services in the API. Streams are not based on the Twitter REST API. For streaming a persistent HTTP connection is opened to the API. This way a client would not have to continuously poll the REST API to register changes or new data. A client would be provided by real time data from Twitter, but only have one single connection opened.~\citep{site:twitterstream}

The Twitter API uses JavaScript Object Notation (JSON) as their format for response. JSON is a lightweight data format often used as an alternative to XML. JSON was created by~\citeauthor{site:json} to be a subset of the JavaScript Programming Language but still be language-independent. JSON has a simple structure and aims to be minimal, portable and textual and has support for four primitive types (strings, numbers, booleans, and null) and two structure types (array and objects).~\citep{site:json}


	
	\subsection{Node.js}
	Node.js is a platform built on Google's V8 JavaScript Engine. In later years Node.js has grown rapidly in popularity, much due to its scalability and event driven nature. As Node.js uses a asynchronous non-blocking I/O model, it is perfect for data-intensive real-time applications~\citep{site:nodejs}. Node.js is often refereed to as JavaScript on the server-side. 

The original goal of Node.js was to enable push capabilities for web sites. That is, for servers to be able to push states to the client, without the client having requested it first. 

Since Node.js is JavaScript based and JSON is a subset of JavaScript, JSON is integrated seamlessly. 

Although Node.js is a fairly new platform, it has a rapid growing user base. Even though Node.js was conceptualized not long ago, and is not yet released as a stable version, there are about 19,000 modules released for it through the Node Package Manager (NPM)~\citep{site:npm}. There are about 700,000 module downloads each day, using NPM. 	
	
	\subsection{Natural Language Toolkit for Python}

\section{Structured Literature Review}
This section will describe what a systematic literature review (SLR) is and how it is conducted. 

A SLR is a methodological and formal way of retrieving information and doing literature search regarding a topic. SLR has been used to a great extent in fields like medicine, but not as much in computer science. An SLR uses several steps to accomplish the literature search and uses a systematic literature review protocol to document these steps. This way the literature search can be reproducible and is documented~\citep{paper:slrdesc}.

\subsection{Performing a SLR}

An SLR consists of three phases with 13 steps in total: planning, conducting and reporting, as defined by \cite{paper:slrdesc} and \cite{master:slr}:

\subsubsection{Planning}

\begin{description}

	\item[1. Identification of the need for a review] \hfill \\
		Identify whether or not the researchers need to review literature using SLR.

	\item[2. Commissioning a review] \hfill \\
		Organizations need to have the resources to do an SLR and commission the review.

	\item[3. Specifying the research question(s)] \hfill \\
		Define research questions (RQ) to specify what the researchers hope to find information about. The RQs are used as basis for, among others, the search, inclusion, and data collection in the conduction phase. 

	\item[4. Developing a review protocol] \hfill \\
		Develop a protocol providing documentation for all steps for conducting an SLR.
	

	\item[5. Evaluating the review protocol] \hfill \\
		Have an independent party or person evaluate the protocol.

\end{description}

% This probably shouldn't be here but in the SLR chapter..
% In this report both step 1 and 2 are assumed completed.

\subsubsection{Conducting}

\begin{description}

	\item[1. Identification of research] \hfill \\
		Locate as many papers as possible relevant to the previously defined and documented RQs. The search strategy needs to be documented with clear definitions for search domains and search string.

	\item[2. Selection of primary studies] \hfill \\
		Use a set of inclusion criteria (IC) to select what papers to use for the review. This can be done in two steps, by filtering on only title and abstract or based on the full text paper. 

	\item[3. Study quality assessment] \hfill \\
		Use a set of quality criteria (QC) to further filter down the paper collection. Use the QCs to set a score of each paper. Set a lower score limit to filter irrelevant or papers of poor quality.

	\item[4. Data extraction and monitoring] \hfill \\
		Extract data from all the papers selected and included. This data collection should be documented by the protocol, defining what features and information to extract.
	

	\item[5. Data synthesis] \hfill \\
		Synthesise the data to be able to answer the RQs. 
\end{description}

% This probably shouldn't be here but in the SLR chapter..
% For this report, the data synthesis is included as a part of the data extraction. 

\subsubsection{Reporting}


\begin{description}

	\item[1. Specifying dissemination strategy] \hfill \\
		Specify how the review result shall be presented. 

	\item[2. Formatting the main report] \hfill \\
		Report the literature review in its own report or as a part of a report.

	\item[3. Evaluating the report] \hfill \\
		Have an independent expert in the field evaluate the SLR. 

\end{description}

