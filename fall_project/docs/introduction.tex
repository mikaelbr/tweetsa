\section{Background and Motivation}
	\subsection{Task Description}

\begin{center} \Large Sentiment Analysis using the Twitter Corpus \end{center}
\begin{normalsize}
In recent years, micro-blogging has become prevalent, and the Twitter API allows users to collect a corpus from their micro-blogosphere. The posts, named tweets are limited to 140 characters, and are often used to express positive or negative emotions to a person or product.

In this project, the goal is to use the Twitter corpus to do sentiment analysis. Pak and Paroubek (2010) have shown how to do this using frameworks like Support Vector Machines (SVMs) and Conditional Random Fields (CRFs), benchmarked with a Naive Bayes Classifier baseline. They were unable to beat the baseline, and the goal of this project will be to experiment with these and other machine learning frameworks as Maximum Entropy learners to try to beat the baseline.
\end{normalsize}

\section{Project goals}
In this section we will describe our goals for this project. As this assignment was intended for a master thesis, we had to scope down the goals to fit the time schedule for a specialization project. We've scoped the assignment down to a set of goals that will make a good foundation for a master thesis.

	\subsection{Research on state-of-the-art sentiment analysis}
	A lot of work has already been done in the field of sentiment analysis, also when using the Twitter corpus. To be able to make a contribution to this work, we have to do research to gather knowledge about existing solutions and their performance. - Mention SLR here -

	\subsection{Server for sentiment classification}
	We will design and implement a highly modular python server with a basic form of classification. This system will work as a foundation for implementing the complete classification system in the proceeding project. For that reason we want it to be as modular as possible, to make it easy to swap out different parts of the system when necessary.

	\subsection{API layer with twitter API integration}
	Twitter offers a well documented REST API to obtain data from their corpus. To make our system easy to use for developers already using the Twitter platform, we will implement a web service that is compatible with the Twitter API. And as an extension to the already existing API we'll add a \emph{sentiment} attribute to the returned tweet. This attribute should hold the result of the sentiment classification.
	
	