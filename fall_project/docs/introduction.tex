\section{Background and Motivation}
	\subsection{Task Description}

\begin{center} \Large Sentiment Analysis using the Twitter Corpus \end{center}
\begin{normalsize}
In recent years, micro-blogging has become prevalent, and the Twitter API allows users to collect a corpus from their micro-blogosphere. The posts, named tweets are limited to 140 characters, and are often used to express positive or negative emotions to a person or product.

In this project, the goal is to use the Twitter corpus to do sentiment analysis. Pak and Paroubek (2010) have shown how to do this using frameworks like Support Vector Machines (SVMs) and Conditional Random Fields (CRFs), benchmarked with a Naive Bayes Classifier baseline. They were unable to beat the baseline, and the goal of this project will be to experiment with these and other machine learning frameworks as Maximum Entropy learners to try to beat the baseline.
\end{normalsize}

