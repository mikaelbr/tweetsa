\section{Task Description}

\begin{center} \Large Sentiment Analysis using the Twitter Corpus \end{center}
\begin{quotation}
In recent years, micro-blogging has become prevalent, and the Twitter API allows users to collect a corpus from their micro-blogosphere. The posts, named tweets are limited to 140 characters, and are often used to express positive or negative emotions to a person or product.

In this project, the goal is to use the Twitter corpus to do sentiment analysis. Pak and Paroubek (2010) have shown how to do this using frameworks like Support Vector Machines (SVMs) and Conditional Random Fields (CRFs), benchmarked with a Naive Bayes Classifier baseline. They were unable to beat the baseline, and the goal of this project will be to experiment with these and other machine learning frameworks as Maximum Entropy learners to try to beat the baseline.
\end{quotation}

\section{Project goals}
In this section the main goals for this project are described. As this assignment was intended for a master thesis, we had to scope down the goals to fit the time schedule for a specialization project.

	\subsection{Research on state-of-the-art sentiment analysis}
	A lot of work has already been done in the field of sentiment analysis, also when using the Twitter corpus. To be able to make a contribution to this work, we have to do research to gather knowledge about existing solutions and their performance.

	\subsection{Server for sentiment classification}
	Design and implement a highly modular python server with a basic form of classification. This system will work as a foundation for implementing the complete classification system in the up-coming master thesis. For that reason it should be as modular as possible, to make it easy to replace different parts of the system when necessary.

	\subsection{API layer with Twitter API integration}
	Twitter offers a well documented Representational State Transfer (REST) Application Programming Interface (API) to obtain data from their corpus. To make our system easy to use for developers already using the Twitter platform, an API that is compatible with the Twitter API should be implemented. As an extension to the output of the already existing API, a \emph{sentiment} attribute will be added to the returned tweet. This attribute should hold the result of the sentiment classification.
	

\section{Twitter}

Twitter has become a popular social media service often referred to as a micro-blogging site. On Twitter users can post messages of 140 characters, called tweets, on their own \emph{timeline}. A timeline is a collection of all user submitted tweets and all tweets from the other users a that user is connected to (following). Tweets can be categorized by using hashtags. A hashtag for instance look like \emph{\#happy} or \emph{\#obama2012}. By annotating the tweets with this tag, users can find similar tweets across Twitter.

Twitter has grown very rapidly and the usage statistics is ever changing. In June 2012, there were posted over 400 million tweets every day~\citep{site:twitterusage}. With over 500 million users, where about 170 million of these are active ones~\citep{site:users}, it is safe to say that Twitter offers a lot of data.


The informal nature of Twitter leads to a lot of sentiments being posted and this has led Twitter to being a gold mine for SA. Many systems have used Twitter as corpus for sentiment analysis. The first one to really use Twitter as a corpus was Sentiment140 (previously known as TwitterSentiment) by Stanford students~\citep{article:go}. After this paper, and as Twitter grew in popularity, many other systems have been developed. Some of the later ones are TwiSent and C-Feel-IT~\citep{mukherjee2012twisent}, Tweenator~\citep{saif2012semantic} and MSAS~\citep{chamlertwat2012discovering}.