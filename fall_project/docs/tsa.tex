This chapter presents some of the state-of-the-art Twitter Sentiment Analysis~(TSA) approaches, and the techniques that are used. The vocabulary used on Twitter makes it hard for traditional natural language processing systems to understand, because they are usually trained on a formal language. This has made researchers exploit some of the special features that the web language, and Twitter gives us, e.g abbreviations and emoticons.
	
		\subsubsection{Data Collection and Preprocessing}
		Most of the data used in TSA research is collected through the Twitter API, either by searching for a certain topic/keyword or by streaming real-time data. Some datasets from related research on the Twitter platform has also been made available for other research projects, as an alternative to collecting a complete data set from scratch. Some approaches specialize on certain domains, while others query for tweets containing emoticons (\emph{':)', ':)'}) to train a cross-domain classifier~\citep{article:go}. The idea behind the emoticon approach is to make sure that the collected tweets contain subjectivity, but these training sets alone are limited to binary classification only (positive/negative classification).

After the data has been collected it should go through a filtering process. First, all non-English tweets are removed, then the Twitter specific symbols and functions described in \autoref{tab:features} would normally be filtered out. As mentioned, a study by \cite{article:go} used ':)' and ':(' as a label for the polarity in their training data, and thus they did not remove these emoticons, but the URLs and usernames were replaced by a nomial ('URL' or 'USERNAME'). They also removed the query term from the text so that it would not affect the classification.

\cite{article:omg} used a hashtagged data set (HASH) in addition to an emoticon data set (EMOT) from http://sentiment140.com. The hashtagged set is a subset of the Edinburgh Twitter corpus which consists of 97 million tweets~\citep{article:edinburgh}.

\begin{table}[]
\centering
\begin{tabular}{|l|l|p{8cm}|}
\hline
RT & Retweet & Reposting another user’s tweet \\ \hline
@ & Mention & Tag used to mention another user \\ \hline
\# & Hashtag & Hashtags are used to tag a tweet to a certain topic. Have become popular recently, and is also used on other platforms \\ \hline
:),:-),$\wedge\wedge$ & Emoticon & Hashtags are used to tag a tweet to a certain topic. Have become popular recently, and is also used on other platforms \\ \hline
URL & URL & Typically a link to an external resource, e.g a new article or a photo \\ \hline
\end{tabular}
\caption{Features that are usually removed from the tweets.}
\label{tab:features}
\end{table}

Some approaches have also experimented with normalizing the tweets, and removing redundant letters, e.g ''loooove'' and ''crazyyy'', that are often used in tweets. Redundant letters like these are sometimes used to express a stronger sentiment, and it has therefore been experimented with trimming down to one additional redundant letter('loooove' = 'loove' instead of love), so that the stronger sentiment can be taken into consideration by a score/weight adjustment for that feature.

\subsubsection*{Part-of-speech tagging}
Part-of-Speech (\nom{POS}{Part-of-Speech}) tagging is a well-known process for marking the words of a sentence. Adjectives, adverbs and personal pronouns have shown good indicators for subjectivity, which has made POS tagging a good technique for filtering out objective tweets before the polarity classification. Early research on TSA showed that the challenging vocabulary made it harder to tag the tweets with a good accuracy; however, in 2010 \cite{article:gimpel} made a POS tagger that aimed at marking tweets. It performed very well in their experiments (almost 90\% accuracy).

		
		\subsubsection{Subjectivity Classification}
		The most used strategy for TSA is a two-step strategy where the first step is subjectivity classification and the second step is the polarity classification. The goal for the subjectivity classification is to separate subjective and objective tweets.

One of the most used techniques for this task is POS tagging. \cite{article:pak} found several indicators of subjectivity by counting word frequencies in a subjective set versus an objective set. They found that interjection and personal pronouns were the strongest indicators of subjectivity in their set. In their paper,~\cite{article:pak} concluded that utterances were a strong indicator of subjectivity, but referring to the tree tag UH. According to the POS guidelines~(\cite{treebank}), tree tag UH is, how ever, not utterances, but rather interjections. In~\cite{article:jiang} used normalization, POS tagging, word stemming and syntactic parsing for the subjectivity classification task. The idea was that normalization of features would give better recall.

Previous research has also explored the use of noisy data and distant supervised methods such as emoticons and hashtags for the subjectivity classification, where any match from a given lexicon will classify the tweet as subjective.

		
		\subsubsection{Polarity Classification}
		The final part of the analysis is the polarity classification (positive/negative). While TSA is not yet considered mature, SA for longer texts, i.e documents and reviews, has been explored for years~\citep{book:pang}. Different techniques and algorithms that have proven worthy for longer texts has also been applied to sentence level SA with various success. Among these techniques, supervised learning methods like naive Bayes classifier (NB), maximum entropy (MaxEnt) and support vector machines (SVM) are the most used. The limited amount of attributes in tweets makes the feature vectors shorter than in documents. For that reason there are no guarantee that algorithms that perform well on document level SA will be the best alternatives for classifying short texts like tweets.

Some approaches have also experimented with a combination of lexicon-based methods and  machine learning~\citep{article:mudinas}. They perform an entity-level sentiment analysis as the first step. Then they use tweets that are likely to be opinionated in a lexicon-based method. The last step of their process is to train a classifier to assign the sentiment value. This approach makes it possible to train the classifier without manually labeling the data, as they’re using the data from the lexicon-based method.\vspace{8mm}

\noindent
\textbf{Supervised learning} \\
\noindent
methods require some sort of training data to create an inferred function for classification tasks. These data would preferably be manually annotated texts, but as this can be a labor-intensive task, some research has experimented with emoticons or a collection of hashtags as labels for positive/negative tweets. This is done by making assumptions, such as all tweets containing positive emoticons are positive, and that all who contain negative emoticon’s are negative.

Among the machine learning algorithms that perform well on TSA are NB, SVN and MaxEnt. While SVN normally beats NB and MaxEnt on longer texts, it seem to have some trouble with outperforming the NB when feature vectors are shorter, e.i shorter texts. \cite{article:bermingham} has shown this in their comparison of SVN and NB for microblogs.\vspace{8 mm}

\noindent
\textbf{Unsupervised learning} \\
\noindent Of the unsupervised methods in TSA, the lexicon-based seem to be the most used approach. This technique requires a lexicon with a sentiment score for each word. When using such lexicons the classifier can look up all the words in the feature vector, e.g a bag of words, and check the sentiment score if the feature exists in the lexicon. Hence it will not need any training beforehand.
	
Popular sentiment lexicons are SentiWordNet and General Inquirer. Some have also made custom extensions of these lexicons that included manually annotated emoticons and hashtags as well as words. \cite{article:afinn} made a sentiment lexicon called AFINN, specialized for Twitter. It contains a lot of words from the vocabulary used in social networks. AFINN supports slang and abbreviations, e.g ‘n00b’, ‘lol’ and ‘wtf’. This lexicon was made as a response to the ANEW lexicon which works better for document level SA since it does not support the Twitter language.